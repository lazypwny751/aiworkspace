% Turev ve Integral Kurallari + Ornek Diferansiyel Denklemler

\section{Temel Turev Kurallari}

Asagida $f,g$ turevlenebilir fonksiyonlar, $c$ sabit ve $x$ gercek degisken olmak uzere temel kurallar verilmiştir.

\subsection{Dogrusal Turev Kurallari}

\paragraph{Kural 1 (Sabitin turevi).}
$\displaystyle \frac{\dint}{\dint x}(c) = 0.$

\paragraph{Ornek.} $\displaystyle \frac{\dint}{\dint x}(5) = 0.$

\paragraph{Kural 2 (Sabit kat sayili turev).}
$f$ turevlenebilir, $c$ sabit olmak uzere
\[
\frac{\dint}{\dint x}(c\,f(x)) = c\,f'(x).
\]

\paragraph{Ornek.} $f(x)=x^2$ olsun. O halde $f'(x)=2x$ ve $3x^2=3f(x)$ olduguna gore
\[
\frac{\dint}{\dint x}(3x^2)=\frac{\dint}{\dint x}(3f(x))=3f'(x)=3\cdot 2x=6x.
\]

\paragraph{Kural 3 (Toplam / farkin turevi).}
$\displaystyle \frac{\dint}{\dint x}(f(x) \pm g(x)) = f'(x) \pm g'(x).$

\paragraph{Ornek.} $f(x)=x^2,\ g(x)=\sin x$ olsun.
\[
\frac{\dint}{\dint x}(x^2+\sin x)=2x+\cos x.
\]

\subsection{Carpim ve Bolum Kurallari}

\paragraph{Kural 4 (Carpim kurali).}
$\displaystyle \frac{\dint}{\dint x}[f(x)g(x)] = f'(x)g(x)+f(x)g'(x).$

\paragraph{Ornek.} $f(x)=x^2,\ g(x)=e^x$ olsun.
\[
\frac{\dint}{\dint x}(x^2 e^x) = 2x e^x + x^2 e^x = e^x(2x+x^2).
\]

\paragraph{Kural 5 (Bolum kurali).}
$\displaystyle \frac{\dint}{\dint x}\Big[\frac{f(x)}{g(x)}\Big] = \frac{f'(x)g(x)-f(x)g'(x)}{[g(x)]^2},\ g(x)\neq 0.$

\paragraph{Ornek.} $f(x)=x,\ g(x)=x^2+1$ olsun.
\[
\frac{\dint}{\dint x}\Big[\frac{x}{x^2+1}\Big]=\frac{1\cdot(x^2+1)-x\cdot 2x}{(x^2+1)^2}=\frac{1-x^2}{(x^2+1)^2}.
\]

\subsection{Zincir Kurali}

\paragraph{Kural 6 (Zincir kurali).}
$y=f(g(x))$ ise
\[
\frac{\dint y}{\dint x}=f'(g(x))\,g'(x).
\]

\paragraph{Ornek.} $y=\sin(x^2)$.
\[
\frac{\dint y}{\dint x}=\cos(x^2)\cdot 2x=2x\cos(x^2).
\]

\subsection{Guc ve Ustel Fonksiyonlarin Turevi}

\paragraph{Kural 7 (Guc kurali).}
$n$ sabit (gercek) olmak uzere
\[
\frac{\dint}{\dint x}(x^n)=n x^{n-1}.
\]

\paragraph{Ornek.} $\displaystyle \frac{\dint}{\dint x}(x^5)=5x^4.$

\paragraph{Kural 8 (Ustel fonksiyon).}
\[
\frac{\dint}{\dint x}(e^x)=e^x, \qquad \frac{\dint}{\dint x}(a^x)=a^x\ln a,\ a>0, a\neq 1.
\]

\paragraph{Ornek.} $\displaystyle \frac{\dint}{\dint x}(2^x)=2^x\ln 2.$

\subsection{Trigonometri ve Logaritma}

\paragraph{Kural 9 (Temel trigonometrik turevler).}
\[
\frac{\dint}{\dint x}(\sin x)=\cos x,\quad
\frac{\dint}{\dint x}(\cos x)=-\sin x,\quad
\frac{\dint}{\dint x}(\tan x)=\sec^2 x.
\]

\paragraph{Ornek.}
\[
\frac{\dint}{\dint x}(\sin x+\cos x)=\cos x-\sin x.
\]

\paragraph{Kural 10 (Logaritma turevi).}
\[
\frac{\dint}{\dint x}(\ln x)=\frac{1}{x},\ x>0,\qquad
\frac{\dint}{\dint x}(\log_a x)=\frac{1}{x\ln a}.
\]

\paragraph{Ornek.} $\displaystyle \frac{\dint}{\dint x}(\ln(x^2+1))=\frac{2x}{x^2+1}.$

\section{Kismi Turevler}

Iki degiskenli bir fonksiyon icin $z=f(x,y)$ kismi turevler:
\[
\frac{\partial f}{\partial x},\qquad \frac{\partial f}{\partial y}.
\]

\paragraph{Kural 11 (Kismi turev tanimi).}
$y$ sabit kabul edilerek $x$'e gore, veya tersi.

\paragraph{Ornek.} $f(x,y)=x^2y+e^{xy}$ icin
\[
\frac{\partial f}{\partial x}=2xy + y e^{xy},\qquad
\frac{\partial f}{\partial y}=x^2 + x e^{xy}.
\]

\paragraph{Kural 12 (Yuksek mertebeden kismi turevler).}
\[
\frac{\partial^2 f}{\partial x^2},\ \frac{\partial^2 f}{\partial y^2},\ \frac{\partial^2 f}{\partial x\partial y} = \frac{\partial}{\partial x}\Big(\frac{\partial f}{\partial y}\Big).
\]

\paragraph{Ornek.} $f(x,y)=x^2y$ icin
\[
\frac{\partial^2 f}{\partial x^2}=2y,\qquad
\frac{\partial^2 f}{\partial x\partial y}=2x.
\]

\section{Belirsiz Integraller}

Belirsiz integral, turevin ters islemidir; $F'(x)=f(x)$ ise
\[
\int f(x)\,\dint x = F(x)+C.
\]

\subsection{Temel Integral Kurallari}

\paragraph{Kural 13 (Sabitin integrali).}
\[
\int c\,\dint x = cx + C.
\]

\paragraph{Ornek.} $\displaystyle \int 5\,\dint x=5x+C.$

\paragraph{Kural 14 (Guc kurali).}
$n\neq -1$ olmak uzere
\[
\int x^n\,\dint x = \frac{x^{n+1}}{n+1}+C.
\]

\paragraph{Ornek.} $\displaystyle \int x^3\,\dint x=\frac{x^4}{4}+C.$

\paragraph{Kural 15 (Dogrusallik).}
\[
\int [af(x)+bg(x)]\,\dint x = a\int f(x)\,\dint x + b\int g(x)\,\dint x.
\]

\paragraph{Ornek.}
\[
\int (2x+3)\,\dint x =2\int x\,\dint x+3\int 1\,\dint x =2\cdot\frac{x^2}{2}+3x+C=x^2+3x+C.
\]

\subsection{Temel Integral Tablolari}

\paragraph{Kural 16 (Ustel ve logaritma).}
\[
\int e^x\,\dint x = e^x + C,\qquad
\int a^x\,\dint x = \frac{a^x}{\ln a}+C,\ a>0, a\neq 1.
\]

\paragraph{Kural 17 (Trigonometri).}
\[
\int \cos x\,\dint x = \sin x + C,\qquad
\int \sin x\,\dint x = -\cos x + C,
\]
\[
\int \sec^2 x\,\dint x = \tan x + C.
\]

\paragraph{Kural 18 (Logaritma).}
\[
\int \frac{1}{x}\,\dint x = \ln|x|+C,\ x\neq 0.
\]

\subsection{Degisken Donusumu ve Kismi Integrasyon}

\paragraph{Kural 19 (Substitusyon / degisken degistirme).}
$u=g(x)$, $g$ tersinir ve turevlenebilir ise
\[
\int f(g(x))g'(x)\,\dint x = \int f(u)\,\dint u.
\]

\paragraph{Ornek.} $\displaystyle \int 2x e^{x^2}\,\dint x$ icin $u=x^2$ alalim.
\[
\int 2x e^{x^2}\,\dint x = \int e^u\,\dint u = e^u+C = e^{x^2}+C.
\]

\paragraph{Kural 20 (Kismi integrasyon).}
\[
\int u\,\dint v = u v - \int v\,\dint u.
\]

\paragraph{Ornek.} $\displaystyle \int x e^x\,\dint x$; $u=x,\ \dint v=e^x\dint x$.
\[
\int x e^x\,\dint x = x e^x - \int e^x\,\dint x = x e^x - e^x + C = e^x(x-1)+C.
\]

\section{Belirli Integraller}

\paragraph{Kural 21 (Tanim).}
\[
\int_a^b f(x)\,\dint x = F(b)-F(a),\quad F'(x)=f(x).
\]

\paragraph{Ornek.} $\displaystyle \int_0^1 x^2\,\dint x$ icin $F(x)=\frac{x^3}{3}$.
\[
\int_0^1 x^2\,\dint x = \Big[\frac{x^3}{3}\Big]_0^1 = \frac{1}{3}-0=\frac{1}{3}.
\]

\paragraph{Kural 22 (Dogrusallik ve bolme).}
\[
\int_a^b [af(x)+bg(x)]\,\dint x = a\int_a^b f(x)\,\dint x + b\int_a^b g(x)\,\dint x,
\]
\[
\int_a^b f(x)\,\dint x = \int_a^c f(x)\,\dint x + \int_c^b f(x)\,\dint x.
\]

\paragraph{Ornek.}
\[
\int_0^2 (x+1)\,\dint x = \int_0^2 x\,\dint x + \int_0^2 1\,\dint x = \Big[\frac{x^2}{2}\Big]_0^2 + [x]_0^2 = 2+2=4.
\]

\paragraph{Kural 23 (Simetri).}
Tek fonksiyon: $f(-x)=-f(x)$, cift fonksiyon: $f(-x)=f(x)$ icin
\[
\int_{-a}^a f(x)\,\dint x = 0\quad (f \text{ tek}),\qquad
\int_{-a}^a f(x)\,\dint x = 2\int_0^a f(x)\,\dint x\quad (f \text{ cift}).
\]

\paragraph{Ornek.} $f(x)=x^3$ (tek), $\displaystyle \int_{-1}^1 x^3\,\dint x=0.$

\section{Katli Integraller}

\subsection{Iki Katli Integral}

\paragraph{Kural 24 (Tanim).} $D$ bolgesi uzerinde
\[
\iint\limits_D f(x,y)\,\dint A = \iint\limits_D f(x,y)\,\dint x\,\dint y.
\]

Dikdortgen bolge: $D=[a,b]\times[c,d]$ ise
\[
\iint\limits_D f(x,y)\,\dint x\,\dint y = \int_c^d \int_a^b f(x,y)\,\dint x\,\dint y = \int_a^b \int_c^d f(x,y)\,\dint y\,\dint x.
\]

\paragraph{Ornek.} $f(x,y)=x+y$, $D=[0,1]\times[0,1]$.
\[
\iint\limits_D (x+y)\,\dint x\,\dint y = \int_0^1 \int_0^1 (x+y)\,\dint x\,\dint y.
\]
Once $x$'e gore:
\[
\int_0^1 (x+y)\,\dint x = \Big[\frac{x^2}{2}+xy\Big]_0^1=\frac{1}{2}+y.
\]
Sonra $y$'ye gore:
\[
\int_0^1 \Big(\frac{1}{2}+y\Big)\,\dint y = \Big[\frac{y}{2}+\frac{y^2}{2}\Big]_0^1 = \frac{1}{2}+\frac{1}{2}=1.
\]

\subsection{Uc Katli Integral}

\paragraph{Kural 25 (Tanim).} $E$ bolgesi icin
\[
\iiint\limits_E f(x,y,z)\,\dint V = \iiint\limits_E f(x,y,z)\,\dint x\,\dint y\,\dint z.
\]

Dikdortgensel paralelkenar bolge: $E=[a,b]\times[c,d]\times[e,f]$ ise
\[
\iiint\limits_E f(x,y,z)\,\dint x\,\dint y\,\dint z = \int_e^f \int_c^d \int_a^b f(x,y,z)\,\dint x\,\dint y\,\dint z.
\]

\paragraph{Ornek.} $f(x,y,z)=x$, $E=[0,1]\times[0,1]\times[0,1]$.
\[
\iiint\limits_E x\,\dint x\,\dint y\,\dint z = \int_0^1 \int_0^1 \int_0^1 x\,\dint x\,\dint y\,\dint z.
\]
Once $x$'e gore:
\[
\int_0^1 x\,\dint x = \Big[\frac{x^2}{2}\Big]_0^1 = \frac{1}{2}.
\]
Sonra sabit $\frac{1}{2}$ icin
\[
\int_0^1 \int_0^1 \frac{1}{2}\,\dint y\,\dint z = \int_0^1 \Big[\frac{y}{2}\Big]_0^1 \dint z = \int_0^1 \frac{1}{2}\,\dint z = \Big[\frac{z}{2}\Big]_0^1 = \frac{1}{2}.
\]

\section{Diferansiyel Denklemler Baglaminda Notlar}

\begin{itemize}[leftmargin=*]
	\item Dogrusal diferansiyel denklemler genellikle turev kurallarinin tersine uygulanmasiyla cozulur; bu nedenle turev ve integral kurallarina hakimiyet kritik onemdedir.
	\item Kismi turevler, ozellikle iki veya daha cok degiskenli diferansiyel denklemlerde (ornegin isi denklemi, dalga denklemi) kullanilir.
	\item Katli integraller, alan/ hacim yorumlari ve bazi diferansiyel denklemlerin integral bicimlerinin hesaplanmasinda sikca karsimiza cikar.
\end{itemize}

\section*{Ornek Diferansiyel Denklem Cozumleri}

Bu bolumde, yukaridaki turev ve integral kurallarini kullanarak iki temel diferansiyel denklemin cozumunu \emph{kural + uygulama} seklinde ozetliyoruz.

\subsection*{Ayrilabilir Denklem Ornegi}

\paragraph{Denklem.} $\displaystyle \frac{\dint x}{\dint y} = \frac{x^3}{y^3}(y-3).$

\paragraph{Adim 1: Degiskenleri ayirma.}
\[
\frac{\dint x}{x^3} = \Big(\frac{y-3}{y^3}\Big)\,\dint y = \Big(\frac{1}{y^2} - \frac{3}{y^3}\Big)\,\dint y.
\]

\paragraph{Adim 2: Her iki tarafi da integrallenme.}
Sol taraf:
\[
\int x^{-3}\,\dint x = -\frac{1}{2x^2} + C_1.
\]
Sag taraf:
\[
\int \Big(\frac{1}{y^2} - \frac{3}{y^3}\Big)\,\dint y = -\frac{1}{y} + \frac{3}{2y^2} + C_2.
\]

\paragraph{Adim 3: Sabitleri birlestirme ve duzenleme.}
Genel sabiti $C$ ile gosterirsek
\[
-\frac{1}{2x^2} = -\frac{1}{y} + \frac{3}{2y^2} + C.
\]
Her iki tarafi $-1$ ile carpalim ve sabiti tekrar isimlendirelim:
\[
\frac{1}{2x^2} = \frac{1}{y} - \frac{3}{2y^2} + C'.
\]
Bu, $x$ ile $y$ arasindaki gizli (implicit) cozum bagintisidir.

Istersek $x$'i acikca yazabiliriz:
\[
\frac{1}{x^2} = \frac{2}{y} - \frac{3}{y^2} + C'' \quad \Rightarrow \quad
x(y) = \pm \frac{1}{\sqrt{\displaystyle \frac{2}{y} - \frac{3}{y^2} + C''}}.
\]

\subsection*{Birinci Mertebe Lineer Denklem Ornegi}

\paragraph{Denklem.} $\displaystyle y' + 2xy = 2x\,e^{-x^2}.$

\paragraph{Adim 1: Standart forma getirme.}
Zaten
\[
y' + P(x)y = Q(x)\quad \text{seklinde},\quad P(x)=2x,\ Q(x)=2x e^{-x^2}.
\]

\paragraph{Adim 2: Entegrasyon faktoru.}
\[
\mu(x) = e^{\int P(x)\,\dint x} = e^{\int 2x\,\dint x} = e^{x^2}.
\]

\paragraph{Adim 3: Tum denklemi $\mu(x)$ ile carpma.}
\[
e^{x^2}y' + 2x e^{x^2}y = 2x.
\]
Sol taraf, carpimin turevi olarak
\[
\frac{\dint}{\dint x}\big(e^{x^2}y\big) = 2x
\]
seklinde yazilabilir.

\paragraph{Adim 4: Integrallenme.}
\[
e^{x^2}y = \int 2x\,\dint x = x^2 + C.
\]
Buradan genel cozum
\[
y(x) = e^{-x^2}(x^2 + C)
\]
olarak elde edilir.

\section*{Ek Turev ve Integral Ornekleri}

Bu bolumde, yukaridaki kurallari pekistirmek icin \emph{cevapsiz} (veya kisaca cevabi verilmis)
kisa ornekler siralanmistir. Sinav oncesi hizli tekrar icin inceleyebilirsin.

\subsection*{Turev Ornekleri (Cozumlu)}

Her sorunun altinda kisa bir cozum veya en azindan sonu\c c verilmistir.

\begin{enumerate}[leftmargin=*]
	\item $f(x) = 3x^4 - 5x^2 + 7$ icin $f'(x)$.
	\\Cozum: Terim terim turev al: $f'(x) = 12x^3 - 10x$.
	\item $f(x) = \sqrt{x} = x^{1/2}$ icin $f'(x)$.
	\\Cozum: Guc kural\i : $f'(x) = \tfrac{1}{2}x^{-1/2} = \dfrac{1}{2\sqrt{x}}$.
	\item $f(x) = \dfrac{1}{x^3} = x^{-3}$ icin $f'(x)$.
	\\Cozum: $f'(x) = -3x^{-4} = -\dfrac{3}{x^4}$.
	\item $f(x) = (2x-1)(x^2+3)$ icin carpim kuralini kullanarak $f'(x)$.
	\\Cozum: $u=2x-1$, $v=x^2+3$. $u'=2$, $v'=2x$.
	\\$f'(x) = u'v + uv' = 2(x^2+3) + (2x-1)2x = 2x^2+6+4x^2-2x = 6x^2-2x+6$.
	\item $f(x) = \dfrac{2x+1}{x^2+1}$ icin bolum kuralini kullanarak $f'(x)$.
	\\Cozum: $u=2x+1$, $v=x^2+1$. $u'=2$, $v'=2x$.
	\\$f'(x) = \dfrac{u'v-u v'}{v^2} = \dfrac{2(x^2+1)-(2x+1)2x}{(x^2+1)^2}
	= \dfrac{2x^2+2-4x^2-2x}{(x^2+1)^2} = \dfrac{-2x^2-2x+2}{(x^2+1)^2}$.
	\item $f(x) = e^{2x}$ icin $f'(x)$.
	\\Cozum: Zincir kural\i : $f'(x) = 2e^{2x}$.
	\item $f(x) = e^{-x^2}$ icin zincir kuraliyla $f'(x)$.
	\\Cozum: $u=-x^2$, $u'=-2x$. $f'(x) = e^{u}u' = -2x e^{-x^2}$.
	\item $f(x) = 2^x$ icin $f'(x)$.
	\\Cozum: $f'(x) = 2^x\ln 2$.
	\item $f(x) = \ln(3x+1)$ icin $f'(x)$.
	\\Cozum: $u=3x+1$, $u'=3$. $f'(x) = \dfrac{u'}{u} = \dfrac{3}{3x+1}$.
	\item $f(x) = \ln(x^2+4)$ icin zincir kuraliyla $f'(x)$.
	\\Cozum: $u=x^2+4$, $u'=2x$. $f'(x) = \dfrac{2x}{x^2+4}$.
	\item $f(x) = \sin(2x)$ icin $f'(x)$.
	\\Cozum: Zincir kural\i : $f'(x) = 2\cos(2x)$.
	\item $f(x) = \cos(3x)$ icin $f'(x)$.
	\\Cozum: $f'(x) = -3\sin(3x)$.
	\item $f(x) = \sin(x^2)$ icin zincir kuralini kullanarak $f'(x)$.
	\\Cozum: $u=x^2$, $u'=2x$. $f'(x) = 2x\cos(x^2)$.
	\item $f(x) = x^2\sin x$ icin carpim kuraliyla $f'(x)$.
	\\Cozum: $u=x^2$, $u'=2x$; $v=\sin x$, $v'=\cos x$.
	\\$f'(x) = u'v + uv' = 2x\sin x + x^2\cos x$.
	\item $f(x) = \dfrac{\sin x}{x}$ icin bolum kuraliyla $f'(x)$.
	\\Cozum: $u=\sin x$, $u'=\cos x$; $v=x$, $v'=1$.
	\\$f'(x) = \dfrac{u'v-uv'}{v^2} = \dfrac{x\cos x-\sin x}{x^2}$.
\end{enumerate}

\subsection*{Integral Ornekleri (Cozumlu)}

\begin{enumerate}[leftmargin=*]
	\item $\displaystyle \int (4x^3 - 2x)\,\dint x$.
	\\Cozum: Guc kural\i : $\int 4x^3\,\dint x = x^4$, $\int -2x\,\dint x = -x^2$.
	\\Sonuc: $x^4 - x^2 + C$.
	\item $\displaystyle \int (3x^2 + 5)\,\dint x$.
	\\Cozum: $\int 3x^2\,\dint x = x^3$, $\int 5\,\dint x = 5x$.
	\\Sonuc: $x^3 + 5x + C$.
	\item $\displaystyle \int x^{1/2}\,\dint x$.
	\\Cozum: $n=\tfrac{1}{2}$ icin: $\dfrac{x^{3/2}}{3/2} = \dfrac{2}{3}x^{3/2} + C$.
	\item $\displaystyle \int \frac{1}{x^2}\,\dint x = \int x^{-2}\,\dint x$.
	\\Cozum: $\dfrac{x^{-1}}{-1} = -\dfrac{1}{x} + C$.
	\item $\displaystyle \int e^x\,\dint x$.
	\\Cozum: Sonuc $e^x + C$.
	\item $\displaystyle \int e^{2x}\,\dint x$.
	\\Cozum: $u=2x$, $\dint u=2\,\dint x$; $\int e^{2x}\,\dint x = \tfrac{1}{2}e^{2x} + C$.
	\item $\displaystyle \int \sin x\,\dint x$.
	\\Cozum: Sonuc $-\cos x + C$.
	\item $\displaystyle \int \cos 3x\,\dint x$.
	\\Cozum: $u=3x$, $\dint u=3\,\dint x$; $\int \cos 3x\,\dint x = \tfrac{1}{3}\sin 3x + C$.
	\item $\displaystyle \int \frac{1}{x}\,\dint x$.
	\\Cozum: Sonuc $\ln|x| + C$.
	\item $\displaystyle \int \frac{2x}{x^2+1}\,\dint x$.
	\\Cozum: $u=x^2+1$, $\dint u=2x\,\dint x$; $\int \dfrac{2x}{x^2+1}\,\dint x = \ln(x^2+1) + C$.
	\item $\displaystyle \int x e^x\,\dint x$ (kismi integrasyon).
	\\Cozum: Yukarida ornek olarak cozuldu: $\int x e^x\,\dint x = e^x(x-1)+C$.
	\item $\displaystyle \int x e^{x^2}\,\dint x$ (substitusyon).
	\\Cozum: $u=x^2$, $\dint u=2x\,\dint x$; $\int x e^{x^2}\,\dint x = \tfrac{1}{2}e^{x^2} + C$.
	\item $\displaystyle \int \frac{1}{\sqrt{x}}\,\dint x = \int x^{-1/2}\,\dint x$.
	\\Cozum: $\dfrac{x^{1/2}}{1/2} = 2\sqrt{x} + C$.
	\item $\displaystyle \int (x^2+1)^2\,\dint x$ (once genislet, sonra integral al).
	\\Cozum: $(x^2+1)^2 = x^4+2x^2+1$.
	\\$\int (x^4+2x^2+1)\,\dint x = \dfrac{x^5}{5} + \dfrac{2x^3}{3} + x + C$.
	\item $\displaystyle \int_0^1 (2x+1)\,\dint x$ (belirli integral).
	\\Cozum: Ilk olarak $F(x) = x^2 + x$.
	\\$\displaystyle \int_0^1 (2x+1)\,\dint x = F(1)-F(0) = (1^2+1)-(0+0) = 2$.
\end{enumerate}

\subsection*{Kismi Turev ve Katli Integral Ornekleri (Cozumlu)}

\begin{enumerate}[leftmargin=*]
	\item $f(x,y) = x^2y + y^3$ icin $\partial f/\partial x$ ve $\partial f/\partial y$.
	\\Cozum: $\dfrac{\partial f}{\partial x} = 2xy$, $\dfrac{\partial f}{\partial y} = x^2 + 3y^2$.
	\item $f(x,y) = e^{xy}$ icin $\partial f/\partial x$ ve $\partial f/\partial y$.
	\\Cozum: $\dfrac{\partial f}{\partial x} = y e^{xy}$, $\dfrac{\partial f}{\partial y} = x e^{xy}$.
	\item $f(x,y) = x^2 + y^2$ icin $\partial^2 f/\partial x^2$ ve $\partial^2 f/\partial x\partial y$.
	\\Cozum: $\dfrac{\partial f}{\partial x} = 2x$, dolayisiyla $\dfrac{\partial^2 f}{\partial x^2} = 2$.
	\\Ayrica $\dfrac{\partial f}{\partial y} = 2y$, buradan $\dfrac{\partial^2 f}{\partial x\partial y} = 0$.
	\item $\displaystyle \int_0^1 \int_0^2 (x+2y)\,\dint y\,\dint x$.
	\\Cozum: Ic integral: $\int_0^2 (x+2y)\,\dint y = [xy + y^2]_0^2 = 2x + 4$.
	\\Dis integral: $\int_0^1 (2x+4)\,\dint x = [x^2+4x]_0^1 = 1+4 = 5$.
	\item $\displaystyle \int_0^1 \int_0^1 (x^2+y^2)\,\dint x\,\dint y$.
	\\Cozum: Ic integral: $\int_0^1 (x^2+y^2)\,\dint x = \left[ \dfrac{x^3}{3} + x y^2 \right]_0^1 = \dfrac{1}{3} + y^2$.
	\\Dis integral: $\int_0^1 \left( \dfrac{1}{3} + y^2 \right) \dint y = \left[ \dfrac{y}{3} + \dfrac{y^3}{3} \right]_0^1 = \dfrac{1}{3} + \dfrac{1}{3} = \dfrac{2}{3}$.
\end{enumerate}

Toplamda bu liste yaklasik 30 civari temel ornekten olusur; her biri icin kurallardan
hangisini kullanman gerektigini gozden gecirmen, turev-integral refleksini guclendirir.
*** End Patch