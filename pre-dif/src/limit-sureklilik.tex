% Limit ve Sureklilik

\section{Limit ve Sureklilik}

Bu bolum, turev ve diferansiyel denklemler oncesi analiz altyapisini hizlica
hatirlatmak icin ozet niteligindedir.

\subsection{Limit Fikri (Gayriresmi)}

Bir $f(x)$ fonksiyonunun $x \to a$ iken limiti $L$ ise, asagidaki orneklerle
bu fikri pekistirebilirsin:

\begin{itemize}[leftmargin=*]
  \item Basit cebirsel fonksiyonlar icin limit genellikle dogrudan yerine koyma ile bulunur.
  \item Paydada $0$ durumu varsa sadeleştirme veya daha ince analiz gerekir.
\end{itemize}

Ardindan kisa alistirmalar:

\begin{enumerate}[leftmargin=*]
  \item $x\to 2$ icin $f(x) = 3x+1$ fonksiyonunun limitini hesapla.\\
      extbf{Sonuc:} $\lim\limits_{x\to 2} (3x+1) = 3\cdot 2+1 = 7$.
  \item $x\to 1$ icin $f(x) = x^2 - 1$ fonksiyonunun limitini bul ve fonksiyonun
    $x=1$'de surekli olup olmadigini yorumla.\\
      extbf{Sonuc:} $\lim\limits_{x\to 1} (x^2-1) = 0$ ve $f(1)=0$ oldugu icin fonksiyon $x=1$'de sureklidir.
  \item $x\to 0$ icin $f(x) = \dfrac{\sin x}{x}$ fonksiyonunun limitini hesapla.\\
      extbf{Sonuc:} $\displaystyle \lim_{x\to 0} \dfrac{\sin x}{x} = 1$.
  \item $x\to 0$ icin $f(x) = \dfrac{1-\cos x}{x^2}$ fonksiyonunun limitini bul.\\
      extbf{Sonuc:} $\displaystyle \lim_{x\to 0} \dfrac{1-\cos x}{x^2} = \dfrac{1}{2}$.
  \item $x\to \infty$ icin $f(x) = \dfrac{2x^2+1}{x^2+3}$ fonksiyonunun limitini hesapla.\\
      extbf{Sonuc:} En baskin terimler oranindan $\displaystyle \lim_{x\to \infty} \dfrac{2x^2+1}{x^2+3} = 2$.
  \item $x\to \infty$ icin $f(x) = \dfrac{3x-5}{2x+7}$ fonksiyonunun limitini bul.\\
      extbf{Sonuc:} $\displaystyle \lim_{x\to \infty} \dfrac{3x-5}{2x+7} = \dfrac{3}{2}$.
  \item $x\to 0$ icin $f(x) = |x|$ fonksiyonunun limitini incele; soldan ve sagdan limitleri
    karsilastir.\\
      extbf{Sonuc:} $\lim\limits_{x\to 0^-} |x| = 0$, $\lim\limits_{x\to 0^+} |x| = 0$ ve $f(0)=0$; limit vardir ve fonksiyon $0$'da sureklidir.
  \item $f(x) = \begin{cases}
  x^2, & x<1, \\
  2x-1, & x\ge 1
    \end{cases}$ icin $x\to 1$ limitini ve $x=1$'de surekliligini incele.\\
      extbf{Sonuc:} Soldan limit $1^2=1$, sagdan limit $2\cdot 1-1=1$ ve $f(1)=1$;
    dolayisiyla $x=1$'de limit vardir ve fonksiyon sureklidir.
  \item $f(x) = \dfrac{x^2-1}{x-1}$ fonksiyonu icin $x\to 1$ limitini hesapla ve fonksiyonun
    $x=1$'de tanimli olup olmadigini yorumla.\\
      extbf{Sonuc:} $\dfrac{x^2-1}{x-1} = x+1$ (\(x\neq 1\) icin), dolayisiyla $\lim\limits_{x\to 1} f(x) = 2$;
    fakat $x=1$ noktasi tanim kumesinde degildir, burada kaldirilabilir bir sureksizlik vardir.
  \item $f(x) = \begin{cases}
  x+1, & x\ne 0, \\
  0, & x=0
    \end{cases}$ fonksiyonu icin $x\to 0$ limitini ve $x=0$'de surekliligini incele.\\
      extbf{Sonuc:} $x\to 0$ icin $x+1\to 1$, yani $\lim\limits_{x\to 0} f(x)=1$;
    ancak $f(0)=0$ oldugu icin fonksiyon $0$'da sureksizdir.
\end{enumerate}

\paragraph{Ders icin mesaj.}
Turev tanimi limit uzerinden yapildigi icin, bu kurallari refleks haline getirmek
hesaplamalari hizlandirir.

\paragraph{Mini ornekler.}
\begin{align*}
  \lim_{x \to 2} (3x^2 - x)
  &= 3\cdot 2^2 - 2 = 12 - 2 = 10, \\
  \lim_{x \to 1} \frac{x^2 - 1}{x-1}
  &= \lim_{x \to 1} (x+1) = 2.
\end{align*}

\subsection{Tek Tarafli ve Sonsuzda Limit}

\begin{itemize}[leftmargin=*]
  \item Sag limit: $x \to a^+$, $x$ degerleri $a$'ya \emph{sagdan} yaklasir.
  \item Sol limit: $x \to a^-$, $x$ degerleri $a$'ya \emph{soldan} yaklasir.
  \item Sag ve sol limit esitse ortak limit vardir.
\end{itemize}

Sonsuzda limitler:
\[
\lim_{x \to \infty} f(x),\qquad \lim_{x \to -\infty} f(x)
\]
fonksiyonun uc noktalardaki davranisini verir.

\paragraph{Ornek.}
\[
\lim_{x \to \infty} \frac{1}{x} = 0.
\]

\paragraph{Mini ornek.}
\[
  \lim_{x \to \infty} \frac{2x^2 + 1}{x^2 - 3}
  = \lim_{x \to \infty} \frac{2 + 1/x^2}{1 - 3/x^2} = 2.
\]

Bu tur dusunceler, ODE cozumlerinin uzun vadede neye yaklastigini (kararlilik,
vanis etme, salinim devam ediyor mu vb.) yorumlarken ise yarar.

\subsection{Sureklilik}

Bir $f$ fonksiyonu $x=a$ noktasinda \emph{s\"urekli} ise:
\begin{enumerate}[leftmargin=*]
  \item $f(a)$ tanimlidir,
  \item $\displaystyle \lim_{x \to a} f(x)$ vardir,
  \item $\displaystyle \lim_{x \to a} f(x) = f(a)$ saglanir.
\end{enumerate}

S\"ureksizlik turleri:
\begin{itemize}[leftmargin=*]
  \item atlamali s\"ureksizlik,
  \item sonsuz s\"ureksizlik,
  \item tanimsiz nokta vb.
\end{itemize}

\paragraph{Mini ornekler.}
\begin{itemize}[leftmargin=*]
  \item $f(x) = \dfrac{1}{x}$ icin $x=0$'da sonsuz s"ureksizlik (dikey asimptot) vardir.
  \item $g(x) = \begin{cases}
          1, & x<0, \\
          2, & x\ge 0
        \end{cases}$ icin $x=0$'da atlamali s"ureksizlik vardir.
\end{itemize}

Polinomlar, ustel fonksiyonlar ve $\sin, \cos$ gibi temel fonksiyonlar
her yerde s\"ureklidir.

\subsection{Turevlenebilirlik ve Sureklilik}

Bir nokta icin:
\begin{itemize}[leftmargin=*]
  \item $f$ o noktada \emph{turevlenebilir} ise, o noktada mutlaka s\"ureklidir.
  \item Tersi her zaman dogru degildir (ornegin $|x|$ fonksiyonunda $x=0$).
\end{itemize}

Turev tanimi:
\[
f'(a) = \lim_{h \to 0} \frac{f(a+h) - f(a)}{h}.
\]

Bu tanimin temelinde hep limit kurallari yatar; pratikte formulleri hazir
kullansan bile, arkadaki teori buraya dayanir.

\subsection{Diferansiyel Denklemlerle Baglanti}

\begin{itemize}[leftmargin=*]
  \item Diferansiyel denklemlerde cozumler genellikle turev ve s\"ureklilik
        varsayimlari uzerine kurulur.
  \item Cozum fonksiyonunun tanim araligini belirlerken limit ve s\"ureklilik
        davranisini dikkate almalisin (ozellikle $\ln x$, kok iceren fonksiyonlar vb.).
\end{itemize}

Bu bolumdeki fikirler, sinavda uzun ispatlar seklinde sorulmasa bile, hangi
fonksiyonlarla rahat calisabilecegini ve nerede dikkatli olman gerektigini
belirlemede sana temel bir sezgi kazandirir.

\section*{Limit ve Sureklilik Ornekleri}

Kisa alistirmalarla limit ve sureklilik kavramlarini pekistirmek icin asagidaki
ornekleri cozebilirsin.

\begin{enumerate}[leftmargin=*]
  \item $\displaystyle \lim_{x \to 2} (x^2 + 3x - 1)$ degerini hesapla.\\
    	extbf{Sonuc:} $x=2$ yazarsan $2^2 + 3\cdot 2 - 1 = 4+6-1 = 9$.
  \item $\displaystyle \lim_{x \to 1} \frac{x^2 - 1}{x-1}$ limitini sadeleştirerek bul.\\
    	extbf{Sonuc:} $\dfrac{x^2-1}{x-1} = x+1$ (\(x\neq 1\) icin), bu yuzden $\lim_{x\to 1} = 2$.
  \item $\displaystyle \lim_{x \to 0} \frac{\sin x}{x}$ icin teorik sonucu hatirla
    (seriler veya grafik uzerinden) ve sonuclandir.\\
      	extbf{Sonuc:} $\displaystyle \lim_{x\to 0} \frac{\sin x}{x} = 1$.
  \item $\displaystyle \lim_{x \to \infty} \frac{3x^2 - x + 1}{x^2 + 4}$ limitini
    pay ve paydayi $x^2$'ye bolerek hesapla.\\
      	extbf{Sonuc:} Baskin terimlerden $\displaystyle \lim_{x\to\infty} \frac{3x^2 - x + 1}{x^2 + 4} = 3$.
  \item $\displaystyle \lim_{x \to 0^+} \ln x$ ve $\displaystyle \lim_{x \to \infty} \ln x$
    limitlerini yorumla.\\
      	extbf{Sonuc:} $x\to 0^+$ icin $\ln x \to -\infty$, $x\to \infty$ icin $\ln x \to \infty$.
  \item
    $f(x) = \begin{cases}
      x^2, & x<1, \\
      2x-1, & x\ge 1
    \end{cases}$ icin $x=1$ noktasinda sag ve sol limitleri, ayrica $f(1)$ degerini hesapla;
    fonksiyonun bu noktada surekli olup olmadigini belirt.\\
      	extbf{Sonuc:} Sol limit $1^2=1$, sag limit $2\cdot 1-1=1$, $f(1)=1$; bu nedenle fonksiyon $x=1$'de sureklidir.
  \item
    $g(x) = \dfrac{1}{x-2}$ fonksiyonu icin $x\to 2^-$ ve $x\to 2^+$ limitlerini incele;
    bu noktada hangi turde bir sureksizlik vardir?\\
      	extbf{Sonuc:} $x\to 2^-$ icin $g(x)\to -\infty$, $x\to 2^+$ icin $g(x)\to +\infty$;
        bu noktada sonsuz (asil) bir sureksizlik vardir.
  \item
  $h(x) = \begin{cases}
      x, & x\ne 0, \\
      1, & x=0
    \end{cases}$ fonksiyonu icin $x=0$'da limit ve fonksiyon degerini karsilastir;
    surekli mi, degil mi?\\
      	extbf{Sonuc:} $\lim_{x\to 0} h(x) = 0$, fakat $h(0)=1$; limit ve deger farkli oldugu icin $x=0$'da sureksizdir.
  \item $\displaystyle \lim_{x \to -\infty} e^x$ ve $\displaystyle \lim_{x \to \infty} e^{-x}$
    limitlerini hesapla; grafiksel yorum yap.\\
      	extbf{Sonuc:} $x\to -\infty$ icin $e^x\to 0$, $x\to \infty$ icin $e^{-x}\to 0$; her iki durumda da eksene yakinlasan ama kesmeyen bir kuyruk vardir.
  \item Bir fonksiyonun bir noktada turevlenebilir olmasi icin hangi sartin once
    saglanmasi gerektigini (sureklilik ile iliskisi) kisa bir cumleyle acikla.\\
      	extbf{Sonuc:} Bir noktada turevlenebilir olmak icin once o noktada surekli olmak gerekir; turevlenebilirlik surekliligi garanti eder ama tersi her zaman dogru degildir.
\end{enumerate}
