% Diferansiyel Denklemler Icin Moduler Notlar
% Ana dosya: src/main.tex

\documentclass[12pt,a4paper]{article}

% Ortak stil (paketler, dil, sayfa ayarlari, komutlar) — template burada
\usepackage{fontspec}
\usepackage{polyglossia}
\setmainlanguage{turkish}
\setotherlanguage{english}
\setmainfont{Latin Modern Roman}

\usepackage{amsmath,amssymb,amsfonts}
\usepackage{geometry}
\geometry{margin=2.5cm}
\usepackage{enumitem}
\usepackage{hyperref}

% SIK KULLANILAN KOMUTLAR
\newcommand{\dint}{\,\mathrm{d}}

\title{Diferansiyel Denklemler Icin Temel Notlar}
\author{Ozet Not Seti}
\date{Kasim 2025}

\begin{document}

\maketitle

\section*{Icindekiler}
\addcontentsline{toc}{section}{Icindekiler}
\tableofcontents
\clearpage

% Giris
\section*{Giris}
\addcontentsline{toc}{section}{Giris}
Bu moduler yapida, diferansiyel denklemler dersinde sik kullanilan temel kavramlar
ayri dosyalar halinde toparlanmistir. Her modul, \emph{kural + ornek} seklinde
kisa ve islevsel bir ozet sunar.

% Buradan itibaren konu bazli dosyalari cekiyoruz

% 1) Polinomlar ve temel fonksiyon tipleri
% Polinomlar ve Temel Fonksiyonlar

\section{Polinomlar ve Temel Fonksiyonlar}

Bu bolum, diferansiyel denklemlerde karsina cikacak fonksiyon tiplerini rahatca
taniman ve onlarla islem yapman icin kisa bir ozet sunar.

\subsection{Polinom Tanim ve Gosterim}

Genel bir polinom
\[
P(x) = a_n x^n + a_{n-1} x^{n-1} + \dots + a_1 x + a_0
\]
seklindedir. Burada
\begin{itemize}[leftmargin=*]
  \item $n$ polinomun \emph{derecesi},
  \item $a_i$'ler \emph{katsayilar},
  \item $a_n \neq 0$ bas katsayidir.
\end{itemize}

\paragraph{Ornek.} $P(x) = 2x^3 - 5x + 1$ icin derece $3$, bas katsayi $2$'dir.

\paragraph{Not.} Diferansiyel denklemlerde sik sik katsayilari polinom olan denklemler
\(y'' - 3y' + 2y = 0\) gibi karsimiza cikar. Katsayilari, dereceleri ve isaretleri
dogru okumak cozum seklini dogru yazmak icin kritiktir.

\subsection{Polinomlarda Temel Islemler}

\begin{itemize}[leftmargin=*]
  \item \textbf{Toplama/\c cikar ma:} Benzer dereceli terimleri topla.
  \item \textbf{Carpma:} Dagilma ozelligi kullanilir; bazen carp anlara ayirmak
        daha pratik olabilir.
  \item \textbf{Bolme:} Uzun bolme veya sentetik bolme; sinavda genelde basit
        ornekler kullanilir.
\end{itemize}

\paragraph{Ornek (toplama).}
\[
(2x^2 + 3x - 1) + (x^2 - x + 4) = 3x^2 + 2x + 3.
\]

\paragraph{Ornek (carpma).}
\[
(x+1)(x-2) = x^2 - x - 2.
\]

Bu temel islemler, ozellikle karakteristik polinomlari cozerken kullanilir;
ornegin
\[
\lambda^2 - 3\lambda + 2 = 0.
\]

\subsection{Kokler ve Carp anlara Ayirma}

Bir polinomun $x=r$ noktasinda \emph{koku} varsa
\[
P(r) = 0
\]
ve \((x-r)\) polinomu \(P(x)\)'i boler.

Iki dereceli denklemler icin standart cozum formulu:
\[
ax^2 + bx + c = 0 \Rightarrow x = \frac{-b \pm \sqrt{b^2 - 4ac}}{2a}.
\]

\paragraph{Ornek.}
\[
x^2 - 3x + 2 = 0 \Rightarrow x=1,\ x=2,
\]
bu nedenle
\[
x^2 - 3x + 2 = (x-1)(x-2).
\]

\paragraph{Diferansiyel denklem baglantisi.}
Iki mertebe sabit katsayili lineer ODE icin karakteristik denklem her zaman bir
polinomdur:
\[
y'' - 3y' + 2y = 0 \Rightarrow \lambda^2 - 3\lambda + 2 = 0 \Rightarrow (\lambda-1)(\lambda-2)=0.
\]
Koklerin turu (gercek, cakisik, karmasik) cozumun seklini belirler; detaylari
daha sonra diferansiyel denklemler bolumunde gorecegiz.

\subsection{Temel Fonksiyon Tipleri}

Diferansiyel denklemlerde sik karsilasan \emph{standart cozum} tipleri:
\begin{enumerate}[leftmargin=*]
  \item Polinomlar: $P(x)$.
  \item Ustel fonksiyonlar: $e^{kx}$, $a^x$.
  \item Trigonometrik fonksiyonlar: $\sin x, \cos x, \tan x, \dots$
  \item Logaritma: $\ln x, \log_a x$.
\end{enumerate}

\paragraph{Neden onemli?}
\begin{itemize}[leftmargin=*]
  \item Sabit katsayili lineer ODE cozumleri genellikle $e^{kx}$, $\sin x$, $\cos x$
        kombinasyonlari olarak ortaya cikar.
  \item Sag tarafta $x^n, e^{ax}, \sin bx, \cos bx$ gibi fonksiyonlar varsa
        ozel cozum tahminleri dogrudan bu form uzerinden yapilir.
\end{itemize}

\subsection{Grafikler ve Temel Sekiller}

\begin{itemize}[leftmargin=*]
  \item Polinomlar genel olarak \emph{puruzsuz} (her yerde turevli) eg riler verir.
  \item $e^x$ her zaman pozitiftir ve $x$ arttikca hizla artar.
  \item $\ln x$ yalnizca $x>0$ icin tanimlidir ve yavas artan bir fonksiyondur.
  \item $\sin x$ ve $\cos x$ periyod ik ve genligi sinirli fonksiyonlardir.
\end{itemize}

Bu sekilleri kafanda canlandirabilmek, c ozumlerin mantikli olup olmadigini
kontrol etmek icin faydalidir.

\subsection{Limit ve Sureklilik ile Baglanti}

\begin{itemize}[leftmargin=*]
  \item Polinomlar her yerde surekli ve turevlenebilirdir.
  \item $e^x$, $\sin x$, $\cos x$ gibi temel fonksiyonlar da ayni ozellige sahiptir.
  \item $\ln x$ yalnizca $x>0$ icin tanimli oldugundan, $x \to 0^+$ davranisi
        (limit) onem kazanir.
\end{itemize}

\paragraph{Ders icin mesaj.}
Bu bolumdeki kavramlar, diferansiyel denklemlerin \emph{on yuzu} gibidir:
karakteristik denklemler, temel cozum tipleri ve cozumun buyuk $x$'lerdeki
veya belirli noktalardaki davranisi icin bu fonksiyonlara hakim olmak gerekir.

\section*{Polinom Ornekleri ve Alistirmalar}

Asagidaki kisa ornekler, polinomlarla hizli islem yapma ve karakteristik
polinom fikrini pekistirmek icin hazirlanmistir.

\begin{enumerate}[leftmargin=*]
  \item $P(x) = 3x^2 - 5x + 2$ polinomunun derecesini, bas katsayisini ve sabit terimini yaz.\\
      extbf{Sonuc:} Derece $2$, bas katsayi $3$, sabit terim $2$.

  \item $P(x) = 2x^3 - x + 1$ icin $P(0)$, $P(1)$ ve $P(-1)$ degerlerini hesapla.\\
      extbf{Sonuc:} $P(0)=1$, $P(1)=2$, $P(-1)=3$.

  \item $P(x) = x^2 - 4x + 3$ polinomunun koklerini bul ve polinomu carp anlara ayir.\\
      extbf{Sonuc:} Kokler $x=1$ ve $x=3$; $P(x) = (x-1)(x-3)$.

  \item $P(x) = x^2 + 5x + 6$ icin kokleri bul; benzer sekilde carp anlara ayir.\\
      extbf{Sonuc:} Kokler $x=2$ ve $x=3$; $P(x) = (x-2)(x-3)$.

  \item $(x+2)(x-3)$ carpimini acip bir polinom olarak yaz; derecesini belirt.\\
      extbf{Sonuc:} $(x+2)(x-3) = x^2-x-6$, derece $2$.

  \item $(2x-1)(x^2+3)$ carpimini ac ve elde ettigin polinomun katsayilarini yaz.\\
      extbf{Sonuc:} $2x^3 - x^2 + 6x - 3$; katsayilar $2, -1, 6, -3$.

  \item Uzun bolme (veya sentetik bolme) kullanarak $x^3 - 1$ polinomunu $(x-1)$'e bol.\\
      extbf{Sonuc:} $x^3-1 = (x-1)(x^2+x+1)$.

  \item $y'' - 5y' + 6y = 0$ icin karakteristik denklemi yaz ve koklerini bul.\\
      extbf{Sonuc:} Karakteristik denklem $\lambda^2 - 5\lambda + 6 = 0$; kokler $\lambda=2$ ve $\lambda=3$.

  \item Bir onceki maddede buldugun koklere gore genel cozumun $y(x)$ seklini tahmin et.\\
      extbf{Sonuc:} $y(x) = C_1 e^{2x} + C_2 e^{3x}$.

  \item $P(x) = x^4 - 1$ polinomunu carp anlara ayir: once ikinci derecedenlere, sonra
    istersen gercek koklere kadar devam et.\\
      extbf{Sonuc:} $x^4-1 = (x^2-1)(x^2+1) = (x-1)(x+1)(x^2+1)$; gercek kokler $x=\pm 1$.
\end{enumerate}


% 2) Temel fonksiyon tipleri ve grafik/sezgi
% Temel Fonksiyonlar ve Grafik Sezgisi Modul Dosyasi

\section{Temel Fonksiyonlar ve Grafik Sezgisi}

Bu bolumde, diferansiyel denklemler dersinde surekli karsilasacagin \emph{temel fonksiyon tipleri}
ve bunlarin grafik/sezgi acisindan davranislari biraz daha detayli ozetlenmistir. Polinomlar bir
onceki bolumde ele alindigi icin, burada daha cok diger sik fonksiyon ailelerine odaklaniyoruz.

\subsection{Us, Kok ve Rasyonel Fonksiyonlar}

\paragraph{Us fonksiyonlari.} \(f(x) = x^n\) seklindeki fonksiyonlar icin:
\begin{itemize}
  \item \(n\) tek ise grafik orijinden gecen S-sekilli (\(x^3\) gibi) veya diagonal (\(x\) gibi) bir sekil cizer.
  \item \(n\) cift ise grafik orijinden gecmez, genelde \(y\)-ekseni etrafinda simetriktir (\(x^2, x^4\) gibi).
\end{itemize}

\paragraph{Mini ornek.} \(f(x)=x^2\) icin \(f(-1)=1\), \(f(0)=0\), \(f(1)=1\); grafik yukari acilan bir paraboldu r.
\(f(x)=x^3\) icin \(f(-1)=-1\), \(f(0)=0\), \(f(1)=1\); grafik orijinden gecen S-seklindedir.

\paragraph{Kok fonksiyonlari.} \(f(x) = \sqrt[m]{x}\) gibi fonksiyonlar, genelde sadece belli
bir aralikta tanimlidir (mesela \(\sqrt{x}\) icin \(x \ge 0\)) ve grafik orijinden cikip yavas
sekilde artar.

\paragraph{Rasyonel fonksiyonlar.} \(f(x) = \dfrac{P(x)}{Q(x)}\) bicimindeki fonksiyonlar, payda
sifir oldugunda tanimsizdir ve bu noktalarda \emph{dikey asimptot} ortaya cikabilir.
Pay ve payda derecelerine gore sonsuzdaki davranis (yatay/asimptotik davranis) belirlenir.

\paragraph{Mini ornek.} \(f(x)=\dfrac{1}{x-1}\) icin \(x=1\) noktasinda tanimsizlik ve dikey asimptot vardir;
\(x\to 1^-\) icin \(f(x)\to -\infty\), \(x\to 1^+\) icin \(f(x)\to +\infty\).

\subsection{Ustel ve Logaritmik Fonksiyonlar}

\paragraph{Ustel fonksiyonlar.} \(f(x) = a^x\) (\(a>0, a\neq 1\)) icin:
\begin{itemize}
  \item Tanım araligi genelde \(\mathbb{R}\), deger araligi \((0,\infty)\) dir.
  \item \(a>1\) ise grafik artan; \(0<a<1\) ise azalan seklindedir.
  \item Diferansiyel denklemlerde, \(e^{kx}\) turu cozumler cok sik karsina cikar.
\end{itemize}

\paragraph{Mini ornek.} \(f(x) = 2^x\) icin \(f(0)=1\), \(f(1)=2\), \(f(2)=4\); \(x\to -\infty\) giderken \(f(x)\to 0\).

\paragraph{Logaritmik fonksiyonlar.} \(f(x) = \log_a x\) icin:
\begin{itemize}
  \item Tanım araligi \((0,\infty)\), deger araligi \(\mathbb{R}\) dir.
  \item \(a>1\) ise artan, \(0<a<1\) ise azalan fonksiyondur.
  \item \(y = \log_a x\), \(y = a^x\)'in ters fonksiyonudur.
\end{itemize}

\paragraph{Mini ornek.} \(f(x)=\ln x\) icin \(f(1)=0\), \(f(e)=1\) ve \(x\to 0^+\) giderken \(f(x)\to -\infty\).

\subsection{Trigonometrik Fonksiyonlar}

\paragraph{Temel trig fonksiyonlari.} \(\sin x, \cos x, \tan x\) ve bunlarin turevleri,
cozduklerin diferansiyel denklemlerde sik cikar:
\begin{itemize}
  \item \(\sin x\) ve \(\cos x\): periyodik, deger araligi \([-1,1]\).
  \item \(\tan x\): periyodik, dikey asimptotlari olan bir fonksiyondur.
  \item \(\sin x\)'in turevi \(\cos x\), \(\cos x\)'in turevi \(-\sin x\)'tir.
\end{itemize}

\paragraph{Mini ornek.} \(f(x)=\sin x\) icin \(f(0)=0\), \(f(\tfrac{\pi}{2})=1\), \(f(\pi)=0\).
Bir tam periyot uzunlugu \(2\pi\)'dir.

Bu fonksiyonlar, salınım (osilasyon) iceren cozumlerde (mesela ikinci mertebe sabit katsayili
lineer ODE'lerde) dogrudan karsina cikar.

\subsection{Grafik ve Sezgi Odakli Notlar}

Farkli fonksiyon tiplerinin grafiklerini kafanda netlestirmek, hem limit/turev yorumunu
hem de diferansiyel denklem cozumlerinin davranisini anlamani cok kolaylastirir.

\begin{itemize}
  \item \textbf{Artan/azalan}: Grafik sag tarafa giderken yukari mi cikiyor, asagi mi iniyor?
  \item \textbf{Konveks/konkav}: Grafikteki "cukur" ve "tepe" hissi, ikinci turev isaretine baglidir.
  \item \textbf{Asimptotlar}: Rasyonel veya logaritmik fonksiyonlarda, grafigin
        yaklastigi ama hic dokunmadigi dogrular var mi?
\end{itemize}

\paragraph{Hizli tekrar icin mini tablo.}
\begin{center}
\begin{tabular}{|c|c|c|c|}
  \hline
  Fonksiyon & Turev & Temel ozellik & Not \\
  \hline
  $x^n$ & $nx^{n-1}$ & polinom, her yerde duzgun & $n$ tamsayi \\
  $e^x$ & $e^x$ & hep pozitif, hizla artan & cozumlerde cok sik \\
  $\ln x$ & $1/x$ & $x>0$, yavas artan & $x\to 0^+$ icin $-\infty$ \\
  $\sin x$ & $\cos x$ & periyodik, sinirli & osilasyon \\
  $\cos x$ & $-\sin x$ & periyodik, sinirli & faz kaymali sin \\
  \hline
\end{tabular}
\end{center}

Bu bolumu, polinomlar ve limit-sureklilik bolumleri ile birlikte dusunebilirsin:
\begin{itemize}
  \item Once \emph{hangi fonksiyonla ugrastigini} tanimla (polinom, us, rasyonel, trig vs.).
  \item Sonra limit/turev/integral sorusunda buna gore davranisi yorumla.
  \item Son olarak, diferansiyel denklem cozumlerinin de bu temel fonksiyonlardan kombinasyonlar
        oldugunu hatirla.
\end{itemize}

\section*{Temel Fonksiyon Ornekleri ve Alistirmalar}

Aşağıdaki kisa sorular, farkli fonksiyon tiplerinin tanim kumesi, degerleri ve temel
grafik davranisini pekistirmek icin hazirlanmistir.

\begin{enumerate}[leftmargin=*]
  \item $f(x) = 2x^2 - 3x + 1$ icin $f(-1)$, $f(0)$ ve $f(2)$ degerlerini hesapla.\\
		extbf{Sonuc:} $f(-1)=2\cdot 1+3+1=6$, $f(0)=1$, $f(2)=2\cdot 4-6+1=3$.

  \item $f(x) = \dfrac{1}{x-2}$ fonksiyonunun tanim kumesini ve $x=2$ yakinindaki
    davranisini (limitleri) yorumla.\\
		extbf{Sonuc:} Tanim kumesi $x\neq 2$; $x\to 2^-$ icin $f(x)\to -\infty$, $x\to 2^+$ icin $f(x)\to +\infty$.

  \item $f(x) = 2^x$ icin $f(0)$, $f(1)$, $f(2)$ ve $f(-1)$ degerlerini bul; $x\to -\infty$
    giderken fonksiyonun neye yaklastigini soyle.\\
		extbf{Sonuc:} $f(0)=1$, $f(1)=2$, $f(2)=4$, $f(-1)=1/2$; $x\to -\infty$ icin $f(x)\to 0$.

  \item $f(x) = 3^x$ fonksiyonunun $x$ arttikca nasil davrandigini kisaca acikla (artan/azalan,
    alt siniri var mi?).\\
		extbf{Sonuc:} $3^x$ artan bir fonksiyondur, alt siniri 0'dir (asla 0 olmaz, 0'a yaklasir).

  \item $f(x) = \ln x$ fonksiyonunun tanim kumesini, $x\to 0^+$ ve $x\to \infty$ icin
    limit davranisini belirt.\\
		extbf{Sonuc:} Tanim kumesi $(0,\infty)$; $x\to 0^+$ icin $\ln x\to -\infty$, $x\to \infty$ icin $\ln x\to \infty$.

  \item $f(x) = \sqrt{x-1}$ icin tanim kumesini bul ve $x$ karsilikli birkac deger icin
    fonksiyon degerlerini hesapla (mesela $x=1,2,5$).\\
		extbf{Sonuc:} Tanim kumesi $x\ge 1$; $f(1)=0$, $f(2)=1$, $f(5)=2$.

  \item $f(x) = \sin x$ icin $x=0, \tfrac{\pi}{2}, \pi, \tfrac{3\pi}{2}$ noktalarinda
    degerleri bul ve bir periyot boyunca grafigin nasil gectigini tarif et.\\
		extbf{Sonuc:} $f(0)=0$, $f(\tfrac{\pi}{2})=1$, $f(\pi)=0$, $f(\tfrac{3\pi}{2})=-1$;
        grafik bir periyotta 0'dan 1'e cikip tekrar 0'a iner ve -1'den yine 0'a doner.

  \item $f(x) = \cos x$ icin bir periyot icerisinde maksimum, minimum ve sifir oldugu
    noktalarin bir listesini yap.\\
		extbf{Sonuc:} Bir periyotta $[0,2\pi]$ icin: maksimum $1$ degeri $x=0,2\pi$'de;
        minimum $-1$ degeri $x=\pi$'de; sifir oldugu noktalar $x=\tfrac{\pi}{2},\tfrac{3\pi}{2}$.

  \item $f(x) = \tan x$ fonksiyonunun tanimsiz oldugu noktalar icin genel bir formul yaz
    (kacinci katlarin \(\tfrac{\pi}{2}\) oldugunu belirt) ve bu noktalarda grafigin
    nasil davrandigini acikla.\\
		extbf{Sonuc:} Tanim siz oldugu noktalar $x = \tfrac{\pi}{2} + k\pi$ (her tamsayi $k$ icin);
        bu noktalarda dikey asimptot vardir, grafik bir tarafta $+\infty$, diger tarafta $-\infty$'ye gider.

  \item Ayni koordinat duzlemine kabaca $y = e^x$ ve $y = \ln x$ grafigini cizdigini hayal et:
    hangi fonksiyon hangi bolgede daha buyuk, nerede kesisiyorlar (yaklasik)?\\
		extbf{Sonuc:} $y=e^x$ her zaman $y=\ln x$'ten daha buyuktur; bunlar aslinda ters fonksiyon
        ciftidir ve $y=x$ dogrusu uzerinde ayna simetrisine sahiptir.
\end{enumerate}


% 3) Limit ve sureklilik
% Limit ve Sureklilik

\section{Limit ve Sureklilik}

Bu bolum, turev ve diferansiyel denklemler oncesi analiz altyapisini hizlica
hatirlatmak icin ozet niteligindedir.

\subsection{Limit Fikri (Gayriresmi)}

Bir $f(x)$ fonksiyonunun $x \to a$ iken limiti $L$ ise, asagidaki orneklerle
bu fikri pekistirebilirsin:

\begin{itemize}[leftmargin=*]
  \item Basit cebirsel fonksiyonlar icin limit genellikle dogrudan yerine koyma ile bulunur.
  \item Paydada $0$ durumu varsa sadeleştirme veya daha ince analiz gerekir.
\end{itemize}

Ardindan kisa alistirmalar:

\begin{enumerate}[leftmargin=*]
  \item $x\to 2$ icin $f(x) = 3x+1$ fonksiyonunun limitini hesapla.\\
      extbf{Sonuc:} $\lim\limits_{x\to 2} (3x+1) = 3\cdot 2+1 = 7$.
  \item $x\to 1$ icin $f(x) = x^2 - 1$ fonksiyonunun limitini bul ve fonksiyonun
    $x=1$'de surekli olup olmadigini yorumla.\\
      extbf{Sonuc:} $\lim\limits_{x\to 1} (x^2-1) = 0$ ve $f(1)=0$ oldugu icin fonksiyon $x=1$'de sureklidir.
  \item $x\to 0$ icin $f(x) = \dfrac{\sin x}{x}$ fonksiyonunun limitini hesapla.\\
      extbf{Sonuc:} $\displaystyle \lim_{x\to 0} \dfrac{\sin x}{x} = 1$.
  \item $x\to 0$ icin $f(x) = \dfrac{1-\cos x}{x^2}$ fonksiyonunun limitini bul.\\
      extbf{Sonuc:} $\displaystyle \lim_{x\to 0} \dfrac{1-\cos x}{x^2} = \dfrac{1}{2}$.
  \item $x\to \infty$ icin $f(x) = \dfrac{2x^2+1}{x^2+3}$ fonksiyonunun limitini hesapla.\\
      extbf{Sonuc:} En baskin terimler oranindan $\displaystyle \lim_{x\to \infty} \dfrac{2x^2+1}{x^2+3} = 2$.
  \item $x\to \infty$ icin $f(x) = \dfrac{3x-5}{2x+7}$ fonksiyonunun limitini bul.\\
      extbf{Sonuc:} $\displaystyle \lim_{x\to \infty} \dfrac{3x-5}{2x+7} = \dfrac{3}{2}$.
  \item $x\to 0$ icin $f(x) = |x|$ fonksiyonunun limitini incele; soldan ve sagdan limitleri
    karsilastir.\\
      extbf{Sonuc:} $\lim\limits_{x\to 0^-} |x| = 0$, $\lim\limits_{x\to 0^+} |x| = 0$ ve $f(0)=0$; limit vardir ve fonksiyon $0$'da sureklidir.
  \item $f(x) = \begin{cases}
  x^2, & x<1, \\
  2x-1, & x\ge 1
    \end{cases}$ icin $x\to 1$ limitini ve $x=1$'de surekliligini incele.\\
      extbf{Sonuc:} Soldan limit $1^2=1$, sagdan limit $2\cdot 1-1=1$ ve $f(1)=1$;
    dolayisiyla $x=1$'de limit vardir ve fonksiyon sureklidir.
  \item $f(x) = \dfrac{x^2-1}{x-1}$ fonksiyonu icin $x\to 1$ limitini hesapla ve fonksiyonun
    $x=1$'de tanimli olup olmadigini yorumla.\\
      extbf{Sonuc:} $\dfrac{x^2-1}{x-1} = x+1$ (\(x\neq 1\) icin), dolayisiyla $\lim\limits_{x\to 1} f(x) = 2$;
    fakat $x=1$ noktasi tanim kumesinde degildir, burada kaldirilabilir bir sureksizlik vardir.
  \item $f(x) = \begin{cases}
  x+1, & x\ne 0, \\
  0, & x=0
    \end{cases}$ fonksiyonu icin $x\to 0$ limitini ve $x=0$'de surekliligini incele.\\
      extbf{Sonuc:} $x\to 0$ icin $x+1\to 1$, yani $\lim\limits_{x\to 0} f(x)=1$;
    ancak $f(0)=0$ oldugu icin fonksiyon $0$'da sureksizdir.
\end{enumerate}

\paragraph{Ders icin mesaj.}
Turev tanimi limit uzerinden yapildigi icin, bu kurallari refleks haline getirmek
hesaplamalari hizlandirir.

\paragraph{Mini ornekler.}
\begin{align*}
  \lim_{x \to 2} (3x^2 - x)
  &= 3\cdot 2^2 - 2 = 12 - 2 = 10, \\
  \lim_{x \to 1} \frac{x^2 - 1}{x-1}
  &= \lim_{x \to 1} (x+1) = 2.
\end{align*}

\subsection{Tek Tarafli ve Sonsuzda Limit}

\begin{itemize}[leftmargin=*]
  \item Sag limit: $x \to a^+$, $x$ degerleri $a$'ya \emph{sagdan} yaklasir.
  \item Sol limit: $x \to a^-$, $x$ degerleri $a$'ya \emph{soldan} yaklasir.
  \item Sag ve sol limit esitse ortak limit vardir.
\end{itemize}

Sonsuzda limitler:
\[
\lim_{x \to \infty} f(x),\qquad \lim_{x \to -\infty} f(x)
\]
fonksiyonun uc noktalardaki davranisini verir.

\paragraph{Ornek.}
\[
\lim_{x \to \infty} \frac{1}{x} = 0.
\]

\paragraph{Mini ornek.}
\[
  \lim_{x \to \infty} \frac{2x^2 + 1}{x^2 - 3}
  = \lim_{x \to \infty} \frac{2 + 1/x^2}{1 - 3/x^2} = 2.
\]

Bu tur dusunceler, ODE cozumlerinin uzun vadede neye yaklastigini (kararlilik,
vanis etme, salinim devam ediyor mu vb.) yorumlarken ise yarar.

\subsection{Sureklilik}

Bir $f$ fonksiyonu $x=a$ noktasinda \emph{s\"urekli} ise:
\begin{enumerate}[leftmargin=*]
  \item $f(a)$ tanimlidir,
  \item $\displaystyle \lim_{x \to a} f(x)$ vardir,
  \item $\displaystyle \lim_{x \to a} f(x) = f(a)$ saglanir.
\end{enumerate}

S\"ureksizlik turleri:
\begin{itemize}[leftmargin=*]
  \item atlamali s\"ureksizlik,
  \item sonsuz s\"ureksizlik,
  \item tanimsiz nokta vb.
\end{itemize}

\paragraph{Mini ornekler.}
\begin{itemize}[leftmargin=*]
  \item $f(x) = \dfrac{1}{x}$ icin $x=0$'da sonsuz s"ureksizlik (dikey asimptot) vardir.
  \item $g(x) = \begin{cases}
          1, & x<0, \\
          2, & x\ge 0
        \end{cases}$ icin $x=0$'da atlamali s"ureksizlik vardir.
\end{itemize}

Polinomlar, ustel fonksiyonlar ve $\sin, \cos$ gibi temel fonksiyonlar
her yerde s\"ureklidir.

\subsection{Turevlenebilirlik ve Sureklilik}

Bir nokta icin:
\begin{itemize}[leftmargin=*]
  \item $f$ o noktada \emph{turevlenebilir} ise, o noktada mutlaka s\"ureklidir.
  \item Tersi her zaman dogru degildir (ornegin $|x|$ fonksiyonunda $x=0$).
\end{itemize}

Turev tanimi:
\[
f'(a) = \lim_{h \to 0} \frac{f(a+h) - f(a)}{h}.
\]

Bu tanimin temelinde hep limit kurallari yatar; pratikte formulleri hazir
kullansan bile, arkadaki teori buraya dayanir.

\subsection{Diferansiyel Denklemlerle Baglanti}

\begin{itemize}[leftmargin=*]
  \item Diferansiyel denklemlerde cozumler genellikle turev ve s\"ureklilik
        varsayimlari uzerine kurulur.
  \item Cozum fonksiyonunun tanim araligini belirlerken limit ve s\"ureklilik
        davranisini dikkate almalisin (ozellikle $\ln x$, kok iceren fonksiyonlar vb.).
\end{itemize}

Bu bolumdeki fikirler, sinavda uzun ispatlar seklinde sorulmasa bile, hangi
fonksiyonlarla rahat calisabilecegini ve nerede dikkatli olman gerektigini
belirlemede sana temel bir sezgi kazandirir.

\section*{Limit ve Sureklilik Ornekleri}

Kisa alistirmalarla limit ve sureklilik kavramlarini pekistirmek icin asagidaki
ornekleri cozebilirsin.

\begin{enumerate}[leftmargin=*]
  \item $\displaystyle \lim_{x \to 2} (x^2 + 3x - 1)$ degerini hesapla.\\
    	extbf{Sonuc:} $x=2$ yazarsan $2^2 + 3\cdot 2 - 1 = 4+6-1 = 9$.
  \item $\displaystyle \lim_{x \to 1} \frac{x^2 - 1}{x-1}$ limitini sadeleştirerek bul.\\
    	extbf{Sonuc:} $\dfrac{x^2-1}{x-1} = x+1$ (\(x\neq 1\) icin), bu yuzden $\lim_{x\to 1} = 2$.
  \item $\displaystyle \lim_{x \to 0} \frac{\sin x}{x}$ icin teorik sonucu hatirla
    (seriler veya grafik uzerinden) ve sonuclandir.\\
      	extbf{Sonuc:} $\displaystyle \lim_{x\to 0} \frac{\sin x}{x} = 1$.
  \item $\displaystyle \lim_{x \to \infty} \frac{3x^2 - x + 1}{x^2 + 4}$ limitini
    pay ve paydayi $x^2$'ye bolerek hesapla.\\
      	extbf{Sonuc:} Baskin terimlerden $\displaystyle \lim_{x\to\infty} \frac{3x^2 - x + 1}{x^2 + 4} = 3$.
  \item $\displaystyle \lim_{x \to 0^+} \ln x$ ve $\displaystyle \lim_{x \to \infty} \ln x$
    limitlerini yorumla.\\
      	extbf{Sonuc:} $x\to 0^+$ icin $\ln x \to -\infty$, $x\to \infty$ icin $\ln x \to \infty$.
  \item
    $f(x) = \begin{cases}
      x^2, & x<1, \\
      2x-1, & x\ge 1
    \end{cases}$ icin $x=1$ noktasinda sag ve sol limitleri, ayrica $f(1)$ degerini hesapla;
    fonksiyonun bu noktada surekli olup olmadigini belirt.\\
      	extbf{Sonuc:} Sol limit $1^2=1$, sag limit $2\cdot 1-1=1$, $f(1)=1$; bu nedenle fonksiyon $x=1$'de sureklidir.
  \item
    $g(x) = \dfrac{1}{x-2}$ fonksiyonu icin $x\to 2^-$ ve $x\to 2^+$ limitlerini incele;
    bu noktada hangi turde bir sureksizlik vardir?\\
      	extbf{Sonuc:} $x\to 2^-$ icin $g(x)\to -\infty$, $x\to 2^+$ icin $g(x)\to +\infty$;
        bu noktada sonsuz (asil) bir sureksizlik vardir.
  \item
  $h(x) = \begin{cases}
      x, & x\ne 0, \\
      1, & x=0
    \end{cases}$ fonksiyonu icin $x=0$'da limit ve fonksiyon degerini karsilastir;
    surekli mi, degil mi?\\
      	extbf{Sonuc:} $\lim_{x\to 0} h(x) = 0$, fakat $h(0)=1$; limit ve deger farkli oldugu icin $x=0$'da sureksizdir.
  \item $\displaystyle \lim_{x \to -\infty} e^x$ ve $\displaystyle \lim_{x \to \infty} e^{-x}$
    limitlerini hesapla; grafiksel yorum yap.\\
      	extbf{Sonuc:} $x\to -\infty$ icin $e^x\to 0$, $x\to \infty$ icin $e^{-x}\to 0$; her iki durumda da eksene yakinlasan ama kesmeyen bir kuyruk vardir.
  \item Bir fonksiyonun bir noktada turevlenebilir olmasi icin hangi sartin once
    saglanmasi gerektigini (sureklilik ile iliskisi) kisa bir cumleyle acikla.\\
      	extbf{Sonuc:} Bir noktada turevlenebilir olmak icin once o noktada surekli olmak gerekir; turevlenebilirlik surekliligi garanti eder ama tersi her zaman dogru degildir.
\end{enumerate}


% 4) Turev ve integral + temel ODE ornekleri
% Turev ve Integral Kurallari + Ornek Diferansiyel Denklemler

\section{Temel Turev Kurallari}

Asagida $f,g$ turevlenebilir fonksiyonlar, $c$ sabit ve $x$ gercek degisken olmak uzere temel kurallar verilmiştir.

\subsection{Dogrusal Turev Kurallari}

\paragraph{Kural 1 (Sabitin turevi).}
$\displaystyle \frac{\dint}{\dint x}(c) = 0.$

\paragraph{Ornek.} $\displaystyle \frac{\dint}{\dint x}(5) = 0.$

\paragraph{Kural 2 (Sabit kat sayili turev).}
$f$ turevlenebilir, $c$ sabit olmak uzere
\[
\frac{\dint}{\dint x}(c\,f(x)) = c\,f'(x).
\]

\paragraph{Ornek.} $f(x)=x^2$ olsun. O halde $f'(x)=2x$ ve $3x^2=3f(x)$ olduguna gore
\[
\frac{\dint}{\dint x}(3x^2)=\frac{\dint}{\dint x}(3f(x))=3f'(x)=3\cdot 2x=6x.
\]

\paragraph{Kural 3 (Toplam / farkin turevi).}
$\displaystyle \frac{\dint}{\dint x}(f(x) \pm g(x)) = f'(x) \pm g'(x).$

\paragraph{Ornek.} $f(x)=x^2,\ g(x)=\sin x$ olsun.
\[
\frac{\dint}{\dint x}(x^2+\sin x)=2x+\cos x.
\]

\subsection{Carpim ve Bolum Kurallari}

\paragraph{Kural 4 (Carpim kurali).}
$\displaystyle \frac{\dint}{\dint x}[f(x)g(x)] = f'(x)g(x)+f(x)g'(x).$

\paragraph{Ornek.} $f(x)=x^2,\ g(x)=e^x$ olsun.
\[
\frac{\dint}{\dint x}(x^2 e^x) = 2x e^x + x^2 e^x = e^x(2x+x^2).
\]

\paragraph{Kural 5 (Bolum kurali).}
$\displaystyle \frac{\dint}{\dint x}\Big[\frac{f(x)}{g(x)}\Big] = \frac{f'(x)g(x)-f(x)g'(x)}{[g(x)]^2},\ g(x)\neq 0.$

\paragraph{Ornek.} $f(x)=x,\ g(x)=x^2+1$ olsun.
\[
\frac{\dint}{\dint x}\Big[\frac{x}{x^2+1}\Big]=\frac{1\cdot(x^2+1)-x\cdot 2x}{(x^2+1)^2}=\frac{1-x^2}{(x^2+1)^2}.
\]

\subsection{Zincir Kurali}

\paragraph{Kural 6 (Zincir kurali).}
$y=f(g(x))$ ise
\[
\frac{\dint y}{\dint x}=f'(g(x))\,g'(x).
\]

\paragraph{Ornek.} $y=\sin(x^2)$.
\[
\frac{\dint y}{\dint x}=\cos(x^2)\cdot 2x=2x\cos(x^2).
\]

\subsection{Guc ve Ustel Fonksiyonlarin Turevi}

\paragraph{Kural 7 (Guc kurali).}
$n$ sabit (gercek) olmak uzere
\[
\frac{\dint}{\dint x}(x^n)=n x^{n-1}.
\]

\paragraph{Ornek.} $\displaystyle \frac{\dint}{\dint x}(x^5)=5x^4.$

\paragraph{Kural 8 (Ustel fonksiyon).}
\[
\frac{\dint}{\dint x}(e^x)=e^x, \qquad \frac{\dint}{\dint x}(a^x)=a^x\ln a,\ a>0, a\neq 1.
\]

\paragraph{Ornek.} $\displaystyle \frac{\dint}{\dint x}(2^x)=2^x\ln 2.$

\subsection{Trigonometri ve Logaritma}

\paragraph{Kural 9 (Temel trigonometrik turevler).}
\[
\frac{\dint}{\dint x}(\sin x)=\cos x,\quad
\frac{\dint}{\dint x}(\cos x)=-\sin x,\quad
\frac{\dint}{\dint x}(\tan x)=\sec^2 x.
\]

\paragraph{Ornek.}
\[
\frac{\dint}{\dint x}(\sin x+\cos x)=\cos x-\sin x.
\]

\paragraph{Kural 10 (Logaritma turevi).}
\[
\frac{\dint}{\dint x}(\ln x)=\frac{1}{x},\ x>0,\qquad
\frac{\dint}{\dint x}(\log_a x)=\frac{1}{x\ln a}.
\]

\paragraph{Ornek.} $\displaystyle \frac{\dint}{\dint x}(\ln(x^2+1))=\frac{2x}{x^2+1}.$

\section{Kismi Turevler}

Iki degiskenli bir fonksiyon icin $z=f(x,y)$ kismi turevler:
\[
\frac{\partial f}{\partial x},\qquad \frac{\partial f}{\partial y}.
\]

\paragraph{Kural 11 (Kismi turev tanimi).}
$y$ sabit kabul edilerek $x$'e gore, veya tersi.

\paragraph{Ornek.} $f(x,y)=x^2y+e^{xy}$ icin
\[
\frac{\partial f}{\partial x}=2xy + y e^{xy},\qquad
\frac{\partial f}{\partial y}=x^2 + x e^{xy}.
\]

\paragraph{Kural 12 (Yuksek mertebeden kismi turevler).}
\[
\frac{\partial^2 f}{\partial x^2},\ \frac{\partial^2 f}{\partial y^2},\ \frac{\partial^2 f}{\partial x\partial y} = \frac{\partial}{\partial x}\Big(\frac{\partial f}{\partial y}\Big).
\]

\paragraph{Ornek.} $f(x,y)=x^2y$ icin
\[
\frac{\partial^2 f}{\partial x^2}=2y,\qquad
\frac{\partial^2 f}{\partial x\partial y}=2x.
\]

\section{Belirsiz Integraller}

Belirsiz integral, turevin ters islemidir; $F'(x)=f(x)$ ise
\[
\int f(x)\,\dint x = F(x)+C.
\]

\subsection{Temel Integral Kurallari}

\paragraph{Kural 13 (Sabitin integrali).}
\[
\int c\,\dint x = cx + C.
\]

\paragraph{Ornek.} $\displaystyle \int 5\,\dint x=5x+C.$

\paragraph{Kural 14 (Guc kurali).}
$n\neq -1$ olmak uzere
\[
\int x^n\,\dint x = \frac{x^{n+1}}{n+1}+C.
\]

\paragraph{Ornek.} $\displaystyle \int x^3\,\dint x=\frac{x^4}{4}+C.$

\paragraph{Kural 15 (Dogrusallik).}
\[
\int [af(x)+bg(x)]\,\dint x = a\int f(x)\,\dint x + b\int g(x)\,\dint x.
\]

\paragraph{Ornek.}
\[
\int (2x+3)\,\dint x =2\int x\,\dint x+3\int 1\,\dint x =2\cdot\frac{x^2}{2}+3x+C=x^2+3x+C.
\]

\subsection{Temel Integral Tablolari}

\paragraph{Kural 16 (Ustel ve logaritma).}
\[
\int e^x\,\dint x = e^x + C,\qquad
\int a^x\,\dint x = \frac{a^x}{\ln a}+C,\ a>0, a\neq 1.
\]

\paragraph{Kural 17 (Trigonometri).}
\[
\int \cos x\,\dint x = \sin x + C,\qquad
\int \sin x\,\dint x = -\cos x + C,
\]
\[
\int \sec^2 x\,\dint x = \tan x + C.
\]

\paragraph{Kural 18 (Logaritma).}
\[
\int \frac{1}{x}\,\dint x = \ln|x|+C,\ x\neq 0.
\]

\subsection{Degisken Donusumu ve Kismi Integrasyon}

\paragraph{Kural 19 (Substitusyon / degisken degistirme).}
$u=g(x)$, $g$ tersinir ve turevlenebilir ise
\[
\int f(g(x))g'(x)\,\dint x = \int f(u)\,\dint u.
\]

\paragraph{Ornek.} $\displaystyle \int 2x e^{x^2}\,\dint x$ icin $u=x^2$ alalim.
\[
\int 2x e^{x^2}\,\dint x = \int e^u\,\dint u = e^u+C = e^{x^2}+C.
\]

\paragraph{Kural 20 (Kismi integrasyon).}
\[
\int u\,\dint v = u v - \int v\,\dint u.
\]

\paragraph{Ornek.} $\displaystyle \int x e^x\,\dint x$; $u=x,\ \dint v=e^x\dint x$.
\[
\int x e^x\,\dint x = x e^x - \int e^x\,\dint x = x e^x - e^x + C = e^x(x-1)+C.
\]

\section{Belirli Integraller}

\paragraph{Kural 21 (Tanim).}
\[
\int_a^b f(x)\,\dint x = F(b)-F(a),\quad F'(x)=f(x).
\]

\paragraph{Ornek.} $\displaystyle \int_0^1 x^2\,\dint x$ icin $F(x)=\frac{x^3}{3}$.
\[
\int_0^1 x^2\,\dint x = \Big[\frac{x^3}{3}\Big]_0^1 = \frac{1}{3}-0=\frac{1}{3}.
\]

\paragraph{Kural 22 (Dogrusallik ve bolme).}
\[
\int_a^b [af(x)+bg(x)]\,\dint x = a\int_a^b f(x)\,\dint x + b\int_a^b g(x)\,\dint x,
\]
\[
\int_a^b f(x)\,\dint x = \int_a^c f(x)\,\dint x + \int_c^b f(x)\,\dint x.
\]

\paragraph{Ornek.}
\[
\int_0^2 (x+1)\,\dint x = \int_0^2 x\,\dint x + \int_0^2 1\,\dint x = \Big[\frac{x^2}{2}\Big]_0^2 + [x]_0^2 = 2+2=4.
\]

\paragraph{Kural 23 (Simetri).}
Tek fonksiyon: $f(-x)=-f(x)$, cift fonksiyon: $f(-x)=f(x)$ icin
\[
\int_{-a}^a f(x)\,\dint x = 0\quad (f \text{ tek}),\qquad
\int_{-a}^a f(x)\,\dint x = 2\int_0^a f(x)\,\dint x\quad (f \text{ cift}).
\]

\paragraph{Ornek.} $f(x)=x^3$ (tek), $\displaystyle \int_{-1}^1 x^3\,\dint x=0.$

\section{Katli Integraller}

\subsection{Iki Katli Integral}

\paragraph{Kural 24 (Tanim).} $D$ bolgesi uzerinde
\[
\iint\limits_D f(x,y)\,\dint A = \iint\limits_D f(x,y)\,\dint x\,\dint y.
\]

Dikdortgen bolge: $D=[a,b]\times[c,d]$ ise
\[
\iint\limits_D f(x,y)\,\dint x\,\dint y = \int_c^d \int_a^b f(x,y)\,\dint x\,\dint y = \int_a^b \int_c^d f(x,y)\,\dint y\,\dint x.
\]

\paragraph{Ornek.} $f(x,y)=x+y$, $D=[0,1]\times[0,1]$.
\[
\iint\limits_D (x+y)\,\dint x\,\dint y = \int_0^1 \int_0^1 (x+y)\,\dint x\,\dint y.
\]
Once $x$'e gore:
\[
\int_0^1 (x+y)\,\dint x = \Big[\frac{x^2}{2}+xy\Big]_0^1=\frac{1}{2}+y.
\]
Sonra $y$'ye gore:
\[
\int_0^1 \Big(\frac{1}{2}+y\Big)\,\dint y = \Big[\frac{y}{2}+\frac{y^2}{2}\Big]_0^1 = \frac{1}{2}+\frac{1}{2}=1.
\]

\subsection{Uc Katli Integral}

\paragraph{Kural 25 (Tanim).} $E$ bolgesi icin
\[
\iiint\limits_E f(x,y,z)\,\dint V = \iiint\limits_E f(x,y,z)\,\dint x\,\dint y\,\dint z.
\]

Dikdortgensel paralelkenar bolge: $E=[a,b]\times[c,d]\times[e,f]$ ise
\[
\iiint\limits_E f(x,y,z)\,\dint x\,\dint y\,\dint z = \int_e^f \int_c^d \int_a^b f(x,y,z)\,\dint x\,\dint y\,\dint z.
\]

\paragraph{Ornek.} $f(x,y,z)=x$, $E=[0,1]\times[0,1]\times[0,1]$.
\[
\iiint\limits_E x\,\dint x\,\dint y\,\dint z = \int_0^1 \int_0^1 \int_0^1 x\,\dint x\,\dint y\,\dint z.
\]
Once $x$'e gore:
\[
\int_0^1 x\,\dint x = \Big[\frac{x^2}{2}\Big]_0^1 = \frac{1}{2}.
\]
Sonra sabit $\frac{1}{2}$ icin
\[
\int_0^1 \int_0^1 \frac{1}{2}\,\dint y\,\dint z = \int_0^1 \Big[\frac{y}{2}\Big]_0^1 \dint z = \int_0^1 \frac{1}{2}\,\dint z = \Big[\frac{z}{2}\Big]_0^1 = \frac{1}{2}.
\]

\section{Diferansiyel Denklemler Baglaminda Notlar}

\begin{itemize}[leftmargin=*]
	\item Dogrusal diferansiyel denklemler genellikle turev kurallarinin tersine uygulanmasiyla cozulur; bu nedenle turev ve integral kurallarina hakimiyet kritik onemdedir.
	\item Kismi turevler, ozellikle iki veya daha cok degiskenli diferansiyel denklemlerde (ornegin isi denklemi, dalga denklemi) kullanilir.
	\item Katli integraller, alan/ hacim yorumlari ve bazi diferansiyel denklemlerin integral bicimlerinin hesaplanmasinda sikca karsimiza cikar.
\end{itemize}

\section*{Ornek Diferansiyel Denklem Cozumleri}

Bu bolumde, yukaridaki turev ve integral kurallarini kullanarak iki temel diferansiyel denklemin cozumunu \emph{kural + uygulama} seklinde ozetliyoruz.

\subsection*{Ayrilabilir Denklem Ornegi}

\paragraph{Denklem.} $\displaystyle \frac{\dint x}{\dint y} = \frac{x^3}{y^3}(y-3).$

\paragraph{Adim 1: Degiskenleri ayirma.}
\[
\frac{\dint x}{x^3} = \Big(\frac{y-3}{y^3}\Big)\,\dint y = \Big(\frac{1}{y^2} - \frac{3}{y^3}\Big)\,\dint y.
\]

\paragraph{Adim 2: Her iki tarafi da integrallenme.}
Sol taraf:
\[
\int x^{-3}\,\dint x = -\frac{1}{2x^2} + C_1.
\]
Sag taraf:
\[
\int \Big(\frac{1}{y^2} - \frac{3}{y^3}\Big)\,\dint y = -\frac{1}{y} + \frac{3}{2y^2} + C_2.
\]

\paragraph{Adim 3: Sabitleri birlestirme ve duzenleme.}
Genel sabiti $C$ ile gosterirsek
\[
-\frac{1}{2x^2} = -\frac{1}{y} + \frac{3}{2y^2} + C.
\]
Her iki tarafi $-1$ ile carpalim ve sabiti tekrar isimlendirelim:
\[
\frac{1}{2x^2} = \frac{1}{y} - \frac{3}{2y^2} + C'.
\]
Bu, $x$ ile $y$ arasindaki gizli (implicit) cozum bagintisidir.

Istersek $x$'i acikca yazabiliriz:
\[
\frac{1}{x^2} = \frac{2}{y} - \frac{3}{y^2} + C'' \quad \Rightarrow \quad
x(y) = \pm \frac{1}{\sqrt{\displaystyle \frac{2}{y} - \frac{3}{y^2} + C''}}.
\]

\subsection*{Birinci Mertebe Lineer Denklem Ornegi}

\paragraph{Denklem.} $\displaystyle y' + 2xy = 2x\,e^{-x^2}.$

\paragraph{Adim 1: Standart forma getirme.}
Zaten
\[
y' + P(x)y = Q(x)\quad \text{seklinde},\quad P(x)=2x,\ Q(x)=2x e^{-x^2}.
\]

\paragraph{Adim 2: Entegrasyon faktoru.}
\[
\mu(x) = e^{\int P(x)\,\dint x} = e^{\int 2x\,\dint x} = e^{x^2}.
\]

\paragraph{Adim 3: Tum denklemi $\mu(x)$ ile carpma.}
\[
e^{x^2}y' + 2x e^{x^2}y = 2x.
\]
Sol taraf, carpimin turevi olarak
\[
\frac{\dint}{\dint x}\big(e^{x^2}y\big) = 2x
\]
seklinde yazilabilir.

\paragraph{Adim 4: Integrallenme.}
\[
e^{x^2}y = \int 2x\,\dint x = x^2 + C.
\]
Buradan genel cozum
\[
y(x) = e^{-x^2}(x^2 + C)
\]
olarak elde edilir.

\section*{Ek Turev ve Integral Ornekleri}

Bu bolumde, yukaridaki kurallari pekistirmek icin \emph{cevapsiz} (veya kisaca cevabi verilmis)
kisa ornekler siralanmistir. Sinav oncesi hizli tekrar icin inceleyebilirsin.

\subsection*{Turev Ornekleri (Cozumlu)}

Her sorunun altinda kisa bir cozum veya en azindan sonu\c c verilmistir.

\begin{enumerate}[leftmargin=*]
	\item $f(x) = 3x^4 - 5x^2 + 7$ icin $f'(x)$.
	\\Cozum: Terim terim turev al: $f'(x) = 12x^3 - 10x$.
	\item $f(x) = \sqrt{x} = x^{1/2}$ icin $f'(x)$.
	\\Cozum: Guc kural\i : $f'(x) = \tfrac{1}{2}x^{-1/2} = \dfrac{1}{2\sqrt{x}}$.
	\item $f(x) = \dfrac{1}{x^3} = x^{-3}$ icin $f'(x)$.
	\\Cozum: $f'(x) = -3x^{-4} = -\dfrac{3}{x^4}$.
	\item $f(x) = (2x-1)(x^2+3)$ icin carpim kuralini kullanarak $f'(x)$.
	\\Cozum: $u=2x-1$, $v=x^2+3$. $u'=2$, $v'=2x$.
	\\$f'(x) = u'v + uv' = 2(x^2+3) + (2x-1)2x = 2x^2+6+4x^2-2x = 6x^2-2x+6$.
	\item $f(x) = \dfrac{2x+1}{x^2+1}$ icin bolum kuralini kullanarak $f'(x)$.
	\\Cozum: $u=2x+1$, $v=x^2+1$. $u'=2$, $v'=2x$.
	\\$f'(x) = \dfrac{u'v-u v'}{v^2} = \dfrac{2(x^2+1)-(2x+1)2x}{(x^2+1)^2}
	= \dfrac{2x^2+2-4x^2-2x}{(x^2+1)^2} = \dfrac{-2x^2-2x+2}{(x^2+1)^2}$.
	\item $f(x) = e^{2x}$ icin $f'(x)$.
	\\Cozum: Zincir kural\i : $f'(x) = 2e^{2x}$.
	\item $f(x) = e^{-x^2}$ icin zincir kuraliyla $f'(x)$.
	\\Cozum: $u=-x^2$, $u'=-2x$. $f'(x) = e^{u}u' = -2x e^{-x^2}$.
	\item $f(x) = 2^x$ icin $f'(x)$.
	\\Cozum: $f'(x) = 2^x\ln 2$.
	\item $f(x) = \ln(3x+1)$ icin $f'(x)$.
	\\Cozum: $u=3x+1$, $u'=3$. $f'(x) = \dfrac{u'}{u} = \dfrac{3}{3x+1}$.
	\item $f(x) = \ln(x^2+4)$ icin zincir kuraliyla $f'(x)$.
	\\Cozum: $u=x^2+4$, $u'=2x$. $f'(x) = \dfrac{2x}{x^2+4}$.
	\item $f(x) = \sin(2x)$ icin $f'(x)$.
	\\Cozum: Zincir kural\i : $f'(x) = 2\cos(2x)$.
	\item $f(x) = \cos(3x)$ icin $f'(x)$.
	\\Cozum: $f'(x) = -3\sin(3x)$.
	\item $f(x) = \sin(x^2)$ icin zincir kuralini kullanarak $f'(x)$.
	\\Cozum: $u=x^2$, $u'=2x$. $f'(x) = 2x\cos(x^2)$.
	\item $f(x) = x^2\sin x$ icin carpim kuraliyla $f'(x)$.
	\\Cozum: $u=x^2$, $u'=2x$; $v=\sin x$, $v'=\cos x$.
	\\$f'(x) = u'v + uv' = 2x\sin x + x^2\cos x$.
	\item $f(x) = \dfrac{\sin x}{x}$ icin bolum kuraliyla $f'(x)$.
	\\Cozum: $u=\sin x$, $u'=\cos x$; $v=x$, $v'=1$.
	\\$f'(x) = \dfrac{u'v-uv'}{v^2} = \dfrac{x\cos x-\sin x}{x^2}$.
\end{enumerate}

\subsection*{Integral Ornekleri (Cozumlu)}

\begin{enumerate}[leftmargin=*]
	\item $\displaystyle \int (4x^3 - 2x)\,\dint x$.
	\\Cozum: Guc kural\i : $\int 4x^3\,\dint x = x^4$, $\int -2x\,\dint x = -x^2$.
	\\Sonuc: $x^4 - x^2 + C$.
	\item $\displaystyle \int (3x^2 + 5)\,\dint x$.
	\\Cozum: $\int 3x^2\,\dint x = x^3$, $\int 5\,\dint x = 5x$.
	\\Sonuc: $x^3 + 5x + C$.
	\item $\displaystyle \int x^{1/2}\,\dint x$.
	\\Cozum: $n=\tfrac{1}{2}$ icin: $\dfrac{x^{3/2}}{3/2} = \dfrac{2}{3}x^{3/2} + C$.
	\item $\displaystyle \int \frac{1}{x^2}\,\dint x = \int x^{-2}\,\dint x$.
	\\Cozum: $\dfrac{x^{-1}}{-1} = -\dfrac{1}{x} + C$.
	\item $\displaystyle \int e^x\,\dint x$.
	\\Cozum: Sonuc $e^x + C$.
	\item $\displaystyle \int e^{2x}\,\dint x$.
	\\Cozum: $u=2x$, $\dint u=2\,\dint x$; $\int e^{2x}\,\dint x = \tfrac{1}{2}e^{2x} + C$.
	\item $\displaystyle \int \sin x\,\dint x$.
	\\Cozum: Sonuc $-\cos x + C$.
	\item $\displaystyle \int \cos 3x\,\dint x$.
	\\Cozum: $u=3x$, $\dint u=3\,\dint x$; $\int \cos 3x\,\dint x = \tfrac{1}{3}\sin 3x + C$.
	\item $\displaystyle \int \frac{1}{x}\,\dint x$.
	\\Cozum: Sonuc $\ln|x| + C$.
	\item $\displaystyle \int \frac{2x}{x^2+1}\,\dint x$.
	\\Cozum: $u=x^2+1$, $\dint u=2x\,\dint x$; $\int \dfrac{2x}{x^2+1}\,\dint x = \ln(x^2+1) + C$.
	\item $\displaystyle \int x e^x\,\dint x$ (kismi integrasyon).
	\\Cozum: Yukarida ornek olarak cozuldu: $\int x e^x\,\dint x = e^x(x-1)+C$.
	\item $\displaystyle \int x e^{x^2}\,\dint x$ (substitusyon).
	\\Cozum: $u=x^2$, $\dint u=2x\,\dint x$; $\int x e^{x^2}\,\dint x = \tfrac{1}{2}e^{x^2} + C$.
	\item $\displaystyle \int \frac{1}{\sqrt{x}}\,\dint x = \int x^{-1/2}\,\dint x$.
	\\Cozum: $\dfrac{x^{1/2}}{1/2} = 2\sqrt{x} + C$.
	\item $\displaystyle \int (x^2+1)^2\,\dint x$ (once genislet, sonra integral al).
	\\Cozum: $(x^2+1)^2 = x^4+2x^2+1$.
	\\$\int (x^4+2x^2+1)\,\dint x = \dfrac{x^5}{5} + \dfrac{2x^3}{3} + x + C$.
	\item $\displaystyle \int_0^1 (2x+1)\,\dint x$ (belirli integral).
	\\Cozum: Ilk olarak $F(x) = x^2 + x$.
	\\$\displaystyle \int_0^1 (2x+1)\,\dint x = F(1)-F(0) = (1^2+1)-(0+0) = 2$.
\end{enumerate}

\subsection*{Kismi Turev ve Katli Integral Ornekleri (Cozumlu)}

\begin{enumerate}[leftmargin=*]
	\item $f(x,y) = x^2y + y^3$ icin $\partial f/\partial x$ ve $\partial f/\partial y$.
	\\Cozum: $\dfrac{\partial f}{\partial x} = 2xy$, $\dfrac{\partial f}{\partial y} = x^2 + 3y^2$.
	\item $f(x,y) = e^{xy}$ icin $\partial f/\partial x$ ve $\partial f/\partial y$.
	\\Cozum: $\dfrac{\partial f}{\partial x} = y e^{xy}$, $\dfrac{\partial f}{\partial y} = x e^{xy}$.
	\item $f(x,y) = x^2 + y^2$ icin $\partial^2 f/\partial x^2$ ve $\partial^2 f/\partial x\partial y$.
	\\Cozum: $\dfrac{\partial f}{\partial x} = 2x$, dolayisiyla $\dfrac{\partial^2 f}{\partial x^2} = 2$.
	\\Ayrica $\dfrac{\partial f}{\partial y} = 2y$, buradan $\dfrac{\partial^2 f}{\partial x\partial y} = 0$.
	\item $\displaystyle \int_0^1 \int_0^2 (x+2y)\,\dint y\,\dint x$.
	\\Cozum: Ic integral: $\int_0^2 (x+2y)\,\dint y = [xy + y^2]_0^2 = 2x + 4$.
	\\Dis integral: $\int_0^1 (2x+4)\,\dint x = [x^2+4x]_0^1 = 1+4 = 5$.
	\item $\displaystyle \int_0^1 \int_0^1 (x^2+y^2)\,\dint x\,\dint y$.
	\\Cozum: Ic integral: $\int_0^1 (x^2+y^2)\,\dint x = \left[ \dfrac{x^3}{3} + x y^2 \right]_0^1 = \dfrac{1}{3} + y^2$.
	\\Dis integral: $\int_0^1 \left( \dfrac{1}{3} + y^2 \right) \dint y = \left[ \dfrac{y}{3} + \dfrac{y^3}{3} \right]_0^1 = \dfrac{1}{3} + \dfrac{1}{3} = \dfrac{2}{3}$.
\end{enumerate}

Toplamda bu liste yaklasik 30 civari temel ornekten olusur; her biri icin kurallardan
hangisini kullanman gerektigini gozden gecirmen, turev-integral refleksini guclendirir.
*** End Patch

% 5) Katli integraller
% Katli Integraller Modul Dosyasi

\section{Katli Integraller}

Bu bolum, iki ve uc katli integralleri \emph{recete gibi} hatirlamak icin kisa bir ozet sunar.

\subsection{Iki Katli Integral: Temel Fikir}

Suresiz ve uygun sekilde duzenli bir \(f(x,y)\) fonksiyonu icin, dikdortgensel bir bolge uzerinde
\(\iint\) su sekilde tanimlanir. \(R = [a,b] \times [c,d]\) olsun. O zaman
\begin{align*}
  \iint_R f(x,y)\,\mathrm{d}A
  &= \int_a^b \int_c^d f(x,y)\,\mathrm{d}y\,\mathrm{d}x \\
  &= \int_c^d \int_a^b f(x,y)\,\mathrm{d}x\,\mathrm{d}y.
\end{align*}

Genelde, hangi siralama kolay geliyorsa onu secersin; ama \emph{sinirlarin dogru yazilmasi} kritik nokta.

\paragraph{Hesaplanmis ornek (dikdortgen bolge).}
\[
  \int_0^1 \int_0^2 (x + y)\,\mathrm{d}y\,\mathrm{d}x.
\]
Once ic integrali (\(y\) ye gore) al:
\begin{align*}
  \int_0^2 (x+y)\,\mathrm{d}y
  &= \left[ xy + \tfrac{1}{2}y^2 \right]_{y=0}^{y=2}
   = 2x + 2.
\end{align*}
Sonra dis integrali (\(x\) e gore) al:
\begin{align*}
  \int_0^1 (2x+2)\,\mathrm{d}x
  &= \left[ x^2 + 2x \right]_0^1 = 1 + 2 = 3.
\end{align*}
Yani iki katli integralin sonucu \(3\)'tur.

\subsection{Dikdortgen Olmayan Bolgeler}

Pratikte siklikla asagidaki tip bolgelerle karsilasirsin:
\begin{itemize}
  \item \(D = \{(x,y) : a \le x \le b,\ g_1(x) \le y \le g_2(x)\}\)
  \item \(D = \{(x,y) : c \le y \le d,\ h_1(y) \le x \le h_2(y)\}\)
\end{itemize}

Ilk tip icin
\[
  \iint_D f(x,y)\,\mathrm{d}A = \int_a^b \left( \int_{g_1(x)}^{g_2(x)} f(x,y)\,\mathrm{d}y \right)\mathrm{d}x,
\]
ikinci tip icin benzer sekilde once \(x\), sonra \(y\) uzerinden integral alirsin.

\paragraph{Pratik recete.}
\begin{enumerate}[label=\arabic*)]
  \item Bolgeyi mutlaka kabaca da olsa ciz.
  \item \(x\) veya \(y\)'ye gore bir \emph{kesit} al: alt ve ust sinir fonksiyonlarini netlestir.
  \item Integral siralamasini bu kesite gore yaz (\(y\) icte, \(x\) diste ya da tam tersi).
\end{enumerate}

\subsection{Koordinat Donusumleri (Ozet)}

Bazi katli integraller, uygun bir koordinat donusumu ile cok daha kolay hale gelir.
En temel ornek \textbf{polar (kutupsal) koordinatlar}dir.

\subsubsection*{Polar koordinatlar}

\[
  x = r \cos\theta, \qquad y = r \sin\theta.
\]

Bu donusumde alan elemani
\[
  \mathrm{d}A = r\,\mathrm{d}r\,\mathrm{d}\theta
\]
seklinde gelir; yani integrale ekstra bir \(r\) carpani eklenir.
Genel form:
\[
  \iint_D f(x,y)\,\mathrm{d}A
  = \int_{\theta_1}^{\theta_2} \int_{r_1(\theta)}^{r_2(\theta)}
      f(r\cos\theta, r\sin\theta)\, r\,\mathrm{d}r\,\mathrm{d}\theta.
\]

Daire, disk, halka gibi dairesel simetrili bolgelerde polar koordinat secimi neredeyse her zaman isin kolaylastirir.

\paragraph{Hesaplanmis ornek (disk uzerinde integral).}

\(D\), merkezde yaricapi 1 olan birim disk olsun: \(x^2 + y^2 \le 1\). \(f(x,y)=x^2 + y^2\)
icin
\[
  \iint_D (x^2 + y^2)\,\mathrm{d}A
\]
degerini hesaplayalim.

Polar koordinatlara gecersek \(x^2 + y^2 = r^2\) ve \(\mathrm{d}A = r\,\mathrm{d}r\,\mathrm{d}\theta\) oldugundan,
bolge \(0 \le r \le 1\), \(0 \le \theta \le 2\pi\) ile tarif edilir:
\begin{align*}
  \iint_D (x^2 + y^2)\,\mathrm{d}A
  &= \int_0^{2\pi} \int_0^1 r^2 \cdot r\,\mathrm{d}r\,\mathrm{d}\theta \\
  &= \int_0^{2\pi} \int_0^1 r^3\,\mathrm{d}r\,\mathrm{d}\theta \\
  &= \int_0^{2\pi} \left[ \tfrac{1}{4} r^4 \right]_0^1 \mathrm{d}\theta \\
  &= \int_0^{2\pi} \tfrac{1}{4}\,\mathrm{d}\theta \\
  &= \tfrac{1}{4} \cdot 2\pi = \frac{\pi}{2}.
\end{align*}

\subsection{Uc Katli Integraller}

Uc degiskenli bir \(f(x,y,z)\) fonksiyonu icin
\[
  \iiint_E f(x,y,z)\,\mathrm{d}V
\]
ifadesi, hacim uzerinde integral anlamina gelir. Dikdortgensel bir bolge icin
\[
  E = [a,b] \times [c,d] \times [e,f]
\]
oldugunda, integral ic ice uc tekli integral seklinde yazilir:
\[
  \iiint_E f(x,y,z)\,\mathrm{d}V
  = \int_a^b \int_c^d \int_e^f f(x,y,z)\,\mathrm{d}z\,\mathrm{d}y\,\mathrm{d}x,
\]
veya degisken siralamasi ihtiyaca gore degistirilebilir.

\subsection{Diferansiyel Denklemlerle Baglanti}

Baslangic diferansiyel denklemleri dersinde katli integraller cok sik sorulmayabilir;
ancak daha ileri konularda ve fiziksel modellere gecince:
\begin{itemize}
  \item Hacim/alan hesaplari,
  \item Yogunluk fonksiyonu ile \emph{toplam kutle},
  \item Ortalama deger hesaplari
\end{itemize}
hep katli integral fikrine dayanir.

Ote yandan, PDE (kismen diferansiyel denklemler) tarafinda sinir deger problemleri,
enerji/isi korunumu gibi konulara girdiginde, bu sayfaya geri donup
\emph{"bolge uzerinde integral nasil yaziliyordu"} diye bakmak isini kolaylastirir.

\section*{Katli Integral Ornekleri ve Alistirmalar}

Asagidaki ornekler, iki katli ve uc katli integrallerin kurulum ve hesaplamasini
pratik etmen icin hazirlanmistir.

\begin{enumerate}[leftmargin=*]
  \item $\displaystyle \int_0^1 \int_0^1 (x + 2y)\,\mathrm{d}y\,\mathrm{d}x$ integralini hesapla.\\
        	extbf{Sonuc:} Ic integral $\left[xy + y^2\right]_0^1 = x+1$;
        dis integral $\int_0^1 (x+1)\,dx = \left[\tfrac{x^2}{2} + x\right]_0^1 = \tfrac{3}{2}$.

  \item $\displaystyle \int_0^2 \int_0^1 (3x - y)\,\mathrm{d}x\,\mathrm{d}y$ icin once ic, sonra dis
    integrali alarak sonucu bul.\\
        	extbf{Sonuc:} Ic integral $\left[\tfrac{3}{2}x^2 - yx\right]_0^2 = 6-2y$;
        dis integral $\int_0^1 (6-2y)\,dy = [6y - y^2]_0^1 = 5$.

  \item $\displaystyle \int_0^1 \int_0^2 (x^2 + y)\,\mathrm{d}y\,\mathrm{d}x$ integralini hesaplarken,
    ister $y$'ye gore ister $x$'e gore once integre etmeyi dene (sonuc ayni olacak).\\
        	extbf{Sonuc:} $\displaystyle \int_0^1 \left[x^2 y + \tfrac{y^2}{2}\right]_0^2 dx = \int_0^1 (2x^2 + 2) dx = \left[\tfrac{2}{3}x^3 + 2x\right]_0^1 = \tfrac{8}{3}$.

  \item $D = [0,1] \times [0,3]$ icin $\displaystyle \iint_D (2x + y)\,\mathrm{d}A$ integralini
    acik integral seklinde yaz ve hesapla.\\
        	extbf{Sonuc:} $\displaystyle \int_0^1 \int_0^3 (2x+y)\,dy\,dx = \int_0^1 (6x+\tfrac{9}{2}) dx
        = \left[3x^2 + \tfrac{9}{2}x\right]_0^1 = \tfrac{15}{2}$.

  \item $D = \{(x,y) : 0 \le x \le 1,\ x \le y \le 1\}$ bolgesi icin
    $\displaystyle \iint_D y\,\mathrm{d}A$ integralini once $y$ sonra $x$ uzerinden yaz.\\
            extbf{Sonuc:} $\displaystyle \int_0^1 \int_x^1 y\,dy\,dx = \int_0^1 \left[\tfrac{y^2}{2}\right]_x^1 dx
        = \int_0^1 \left(\tfrac{1}{2} - \tfrac{x^2}{2}\right) dx = \left[\tfrac{x}{2} - \tfrac{x^3}{6}\right]_0^1 = \tfrac{1}{3}$.

  \item Yukaridaki $D$ bolgesi icin ayni integrali bu kez once $x$, sonra $y$ uzerinden
    yazmayi dene (limitleri degisken fonksiyonlar seklinde ifade et).\\
          extbf{Sonuc:} $D$ icin $0 \le y \le 1$, $0 \le x \le y$; integral $\displaystyle \int_0^1 \int_0^y y\,dx\,dy
  = \int_0^1 y^2 \,dy = \left[\tfrac{y^3}{3}\right]_0^1 = \tfrac{1}{3}$ (ayni sonuc).

  \item Birim kare $[0,1]\times[0,1]$ uzerinde $f(x,y)=x^2y$ icin ortalama deger
    $\displaystyle f_{ort} = \frac{1}{\text{alan}} \iint f\,\mathrm{d}A$ formuluyle hesapla.\\
          extbf{Sonuc:} Alan 1 oldugu icin $f_{ort} = \displaystyle \int_0^1\int_0^1 x^2 y\,dy\,dx
  = \int_0^1 x^2 \left[\tfrac{y^2}{2}\right]_0^1 dx = \int_0^1 \tfrac{x^2}{2} \,dx = \tfrac{1}{6}$.

  \item $E = [0,1]\times[0,1]\times[0,1]$ icin
    $\displaystyle \iiint_E (x + y + z)\,\mathrm{d}V$ hacim integralini hesapla.\\
          extbf{Sonuc:} Ayrilabilir: $\displaystyle \iiint_E (x+y+z)\,dV = \int_0^1 x\,dx + \int_0^1 y\,dy + \int_0^1 z\,dz
  = 3 \cdot \tfrac{1}{2} = \tfrac{3}{2}$.

  \item Polar koordinatlara gecerek, yaricapi 2 olan diskte
    $\displaystyle \iint_D x^2 \,\mathrm{d}A$ integralini kur (hesaplamayi istersen tam
    yap, istersen kurulmus sekilde birak).\\
          extbf{Sonuc:} $x = r\cos\theta$, $\mathrm{d}A = r\,dr\,d\theta$; integrand $x^2 = r^2 \cos^2\theta$;
  integral $\displaystyle \int_0^{2\pi} \int_0^2 r^3 \cos^2\theta\,dr\,d\theta$ seklinde yazilir.

  \item Yine polar koordinatlarla, $\displaystyle \iint_D 1\,\mathrm{d}A$ integralinin aslinda
    bolgenin alanini verdigini hatirla ve yaricapi $R$ olan bir diskin alanini bu sekilde bul.\\
          extbf{Sonuc:} $\displaystyle \int_0^{2\pi} \int_0^R 1\cdot r\,dr\,d\theta = \int_0^{2\pi} \left[\tfrac{r^2}{2}\right]_0^R d\theta
  = \int_0^{2\pi} \tfrac{R^2}{2} \,d\theta = \pi R^2$.
\end{enumerate}


% 6) Diferansiyel denklemler ozet
% Diferansiyel Denklemler Ozet

\section{Diferansiyel Denklemler Ozet}

Bu bolum, sinavda en cok isine yarayacak temel ODE turleri icin kisa bir
\emph{recete} sunar.

\subsection{Temel Tanimlar}

\begin{itemize}[leftmargin=*]
  \item \textbf{Mertebe (order):} Denklemde gecen en yuksek turevin derecesi.
    Ornek: $y'' + 3y' - 2y = 0$ \,$\Rightarrow$\, 2. mertebe.
  \item \textbf{Lineer ODE:} $y, y', y'', \dots$ terimleri 1. dereceden, kendi
    aralarinda carpim yok. Ornek: $y' + p(x)y = q(x)$ lineer; $y^2 y' = x$ lineer degil.
  \item \textbf{Genel cozum:} Icinde sabitler ($C, C_1, C_2, \dots$) bulunan cozum ailesi.
  \item \textbf{Ozel cozum:} Baslangic/kenar kosullari yerlestirince elde edilen belirli cozum.
\end{itemize}

\subsection{Ayrilabilir Denklemler}

\textbf{Form:}
\[
\frac{dy}{dx} = f(x) g(y)
\]
veya buna denk bir yazilis.

\textbf{Recete:}
\begin{enumerate}[leftmargin=*]
  \item Tum $y$ terimlerini bir tarafa, $x$ terimlerini diger tarafa topla:
    \[
    \frac{1}{g(y)} \, dy = f(x) \, dx.
    \]
  \item Her iki tarafi da integre et:
    \[
    \int \frac{1}{g(y)} \, dy = \int f(x) \, dx + C.
    \]
  \item Mumkunse $y$'yi yalniz birak.
\end{enumerate}

\paragraph{Mini ornek.}
\[
\frac{dy}{dx} = x y
\]
Ayir:
\[
\frac{1}{y} \, dy = x \, dx.
\]
Integralle:
\[
\ln|y| = \frac{x^2}{2} + C \quad \Rightarrow \quad y = C e^{x^2/2}.
\]

Bu tip denklemler, temel seviye fizik ve uygulama sorularinda sik karsina cikar.

\subsection{Birinci Mertebe Lineer Denklemler}

\textbf{Genel form:}
\[
y' + p(x) y = q(x).
\]

\textbf{Recete (entegrasyon faktoru):}
\begin{enumerate}[leftmargin=*]
  \item $p(x)$'i belirle.
  \item Entegrasyon faktorunu hesapla:
    \[
    \mu(x) = e^{\int p(x) \, dx}.
    \]
  \item Denklemin her iki yanini $\mu(x)$ ile carp:
    \[
    \mu(x) y' + \mu(x) p(x) y = \mu(x) q(x).
    \]
  \item Sol taraf artik bir turevdir:
    \[
    (\mu(x) y)' = \mu(x) q(x).
    \]
  \item Her iki tarafi integra et:
    \[
    \mu(x) y = \int \mu(x) q(x) \, dx + C.
    \]
  \item Son olarak $y$'yi yalniz birak:
    \[
    y(x) = \frac{1}{\mu(x)}\left( \int \mu(x) q(x) \, dx + C \right).
    \]
\end{enumerate}

\paragraph{Not.} Ana dosyanin turev-integral bolumunde bu tipe ait detayli bir ornek var.

\subsection{Sabit Katsayili 2. Mertebe Lineer ODE'ler}

\textbf{Genel form:}
\[
ay'' + by' + cy = 0, \quad a \ne 0.
\]

\textbf{Adim 1: Karakteristik denklem.}
\[
a\lambda^2 + b\lambda + c = 0.
\]

Cozumler kok turune gore ayrilir.

\subsubsection{Iki Gercek Ayri k Kok (iki farkli gercek kok)}

Karakteristik denklemden iki farkli gercek kok elde edildigini varsayalim; bunlari
$\lambda_1$ ve $\lambda_2$ ile gosterirsek genel cozum
\[
y(x) = C_1 e^{\lambda_1 x} + C_2 e^{\lambda_2 x}
\]
seklindedir.

\subsubsection{Cakisik Gercek Kok (cakisik gercek kok)}

Eger karakteristik denklemden tek bir gercek kok (cakisik kok) cikarsa, bu koku
$\lambda$ ile gosteririz ve genel cozum
\[
y(x) = (C_1 + C_2 x) e^{\lambda x}
\]
seklindedir.

\subsubsection{Karmasik Esl enik Kokler (alfa artı-eksi i beta turu karmasik kokler)}

Karakteristik denklemden karmasik eslenik kokler elde edersek, bunlari
$\lambda_{1,2} = \alpha \pm i\beta$ seklinde yazabiliriz. Bu durumda genel cozum
\[
y(x) = e^{\alpha x} (C_1 \cos(\beta x) + C_2 \sin(\beta x))
\]
bi\c cimindedir.

\paragraph{Ozet.}
\begin{itemize}[leftmargin=*]
  \item $\Delta = b^2 - 4ac > 0$ ise: iki gercek ayri k kok.
  \item $\Delta = 0$ ise: cakisik kok.
  \item $\Delta < 0$ ise: karmasik kokler.
\end{itemize}

\subsection{Zorlanmis (Sag Tarafli) Sabit Katsayili Lineer ODE}

\textbf{Form:}
\[
ay'' + by' + cy = g(x).
\]

\textbf{Genel strateji:}
\begin{enumerate}[leftmargin=*]
  \item Once homojen kismi coz: $ay'' + by' + cy = 0$ \,$\Rightarrow$\, $y_h$.
  \item Sonra, $g(x)$'in turune gore bir \emph{ozel cozum} $y_p$ tahmin et:
    \begin{itemize}[leftmargin=*]
      \item $g(x)$ polinom ise: polinom formunda dene.
      \item $g(x) = e^{kx}$ ise: $Ae^{kx}$ dene.
      \item $g(x) = \sin bx$ veya $\cos bx$ ise: $A\cos bx + B\sin bx$ dene.
    \end{itemize}
  \item Toplam cozum: $y = y_h + y_p$.
\end{enumerate}

Detayli hesap, ders seviyene gore degisebilir; fakat bu \emph{iskelet} hep aynidir.

\subsection{Fiziksel Yorumlar (Cok Kisa)}

\begin{itemize}[leftmargin=*]
  \item \textbf{Harmonik osilator:} $y'' + \omega^2 y = 0$  $\Rightarrow$ sin\"us-kosin\"us tipinde salinim cozumleri.
  \item \textbf{Sonumlu salinim:} $y'' + 2\gamma y' + \omega^2 y = 0$  
    icin koklerin dogasi (gercek/cakisik/karmasik) sistemin ne kadar hizli sonumlend igini belirler.
\end{itemize}

Bu tur yorumlar, elde ettigin cozumun formunun fiziksel olarak mantikli olup
olmadigini kontrol etmek icin iyi bir sezgi kaynagi saglar.

\section*{Diferansiyel Denklem Ornekleri ve Alistirmalar}

Asagidaki kisa ornekler, farkli ODE turleri icin receteleri uygulama alistirmasi
olarak dusunulebilir. Cogu icin sadece genel cozumu bulman yeterlidir.

\begin{enumerate}[leftmargin=*]
  \item $\displaystyle \frac{dy}{dx} = 3y$ icin ayrilabilirlik recetesini kullanarak genel cozumu bul.
  \item $\displaystyle \frac{dy}{dx} = x y$ denklemi icin ayirma adimlarini yaz ve genel cozumu elde et.
  \item $\displaystyle y' + y = 0$ denklemini birinci mertebe lineer receteyle coz.
  \item $\displaystyle y' - 2y = e^{3x}$ denklemi icin integrasyon faktorunu yaz ve genel cozumu bul.
    \textbf{Cozum:} Entegrasyon faktoru $\mu(x) = e^{\int -2 dx} = e^{-2x}$.
    Denklemi carparsak: $(e^{-2x}y)' = e^{-2x}e^{3x} = e^x$.
    Integral alirsak: $e^{-2x}y = \int e^x dx = e^x + C$.
    Genel cozum: $y(x) = e^{3x} + Ce^{2x}$.
  \item $y'' - 3y' + 2y = 0$ icin karakteristik denklemi yaz, kokleri bul ve genel cozumu belirt.
  \item $y'' + y = 0$ denklemi icin cozumun sin ve cos cinsinden genel formunu yaz.
  \item $y'' + 4y = 0$ denkleminin karakteristik koklerini ve buna karsilik gelen salinim
    frekansini (omega) yorumla.
  \item $y'' - y = e^x$ gibi zorlanmis bir denklem icin (homojen + ozel cozum) genel cozum
    stratejisini kisa maddeler halinde ozetle (detayli hesap zorunlu degil).
  \item Baslangic deger problemi: $y' = 2y$, $y(0) = 3$ icin once genel cozumu bul, sonra baslangic
    kosulunu kullanarak sabiti belirle.
  \item $y'' + 2y' + y = 0$ denklemi icin $\Delta = b^2 - 4ac$ degerini hesapla ve koklerin hangi
    tipte oldugunu (ayrik, cakisik, karmasik) siniflandir; buna gore genel cozumu yaz.
\end{enumerate}


% Ileride eklenecek diger bolumler icin (polinomlar, limit, katli integral, ODE ozetleri)
% ornegin:
% % Polinomlar ve Temel Fonksiyonlar

\section{Polinomlar ve Temel Fonksiyonlar}

Bu bolum, diferansiyel denklemlerde karsina cikacak fonksiyon tiplerini rahatca
taniman ve onlarla islem yapman icin kisa bir ozet sunar.

\subsection{Polinom Tanim ve Gosterim}

Genel bir polinom
\[
P(x) = a_n x^n + a_{n-1} x^{n-1} + \dots + a_1 x + a_0
\]
seklindedir. Burada
\begin{itemize}[leftmargin=*]
  \item $n$ polinomun \emph{derecesi},
  \item $a_i$'ler \emph{katsayilar},
  \item $a_n \neq 0$ bas katsayidir.
\end{itemize}

\paragraph{Ornek.} $P(x) = 2x^3 - 5x + 1$ icin derece $3$, bas katsayi $2$'dir.

\paragraph{Not.} Diferansiyel denklemlerde sik sik katsayilari polinom olan denklemler
\(y'' - 3y' + 2y = 0\) gibi karsimiza cikar. Katsayilari, dereceleri ve isaretleri
dogru okumak cozum seklini dogru yazmak icin kritiktir.

\subsection{Polinomlarda Temel Islemler}

\begin{itemize}[leftmargin=*]
  \item \textbf{Toplama/\c cikar ma:} Benzer dereceli terimleri topla.
  \item \textbf{Carpma:} Dagilma ozelligi kullanilir; bazen carp anlara ayirmak
        daha pratik olabilir.
  \item \textbf{Bolme:} Uzun bolme veya sentetik bolme; sinavda genelde basit
        ornekler kullanilir.
\end{itemize}

\paragraph{Ornek (toplama).}
\[
(2x^2 + 3x - 1) + (x^2 - x + 4) = 3x^2 + 2x + 3.
\]

\paragraph{Ornek (carpma).}
\[
(x+1)(x-2) = x^2 - x - 2.
\]

Bu temel islemler, ozellikle karakteristik polinomlari cozerken kullanilir;
ornegin
\[
\lambda^2 - 3\lambda + 2 = 0.
\]

\subsection{Kokler ve Carp anlara Ayirma}

Bir polinomun $x=r$ noktasinda \emph{koku} varsa
\[
P(r) = 0
\]
ve \((x-r)\) polinomu \(P(x)\)'i boler.

Iki dereceli denklemler icin standart cozum formulu:
\[
ax^2 + bx + c = 0 \Rightarrow x = \frac{-b \pm \sqrt{b^2 - 4ac}}{2a}.
\]

\paragraph{Ornek.}
\[
x^2 - 3x + 2 = 0 \Rightarrow x=1,\ x=2,
\]
bu nedenle
\[
x^2 - 3x + 2 = (x-1)(x-2).
\]

\paragraph{Diferansiyel denklem baglantisi.}
Iki mertebe sabit katsayili lineer ODE icin karakteristik denklem her zaman bir
polinomdur:
\[
y'' - 3y' + 2y = 0 \Rightarrow \lambda^2 - 3\lambda + 2 = 0 \Rightarrow (\lambda-1)(\lambda-2)=0.
\]
Koklerin turu (gercek, cakisik, karmasik) cozumun seklini belirler; detaylari
daha sonra diferansiyel denklemler bolumunde gorecegiz.

\subsection{Temel Fonksiyon Tipleri}

Diferansiyel denklemlerde sik karsilasan \emph{standart cozum} tipleri:
\begin{enumerate}[leftmargin=*]
  \item Polinomlar: $P(x)$.
  \item Ustel fonksiyonlar: $e^{kx}$, $a^x$.
  \item Trigonometrik fonksiyonlar: $\sin x, \cos x, \tan x, \dots$
  \item Logaritma: $\ln x, \log_a x$.
\end{enumerate}

\paragraph{Neden onemli?}
\begin{itemize}[leftmargin=*]
  \item Sabit katsayili lineer ODE cozumleri genellikle $e^{kx}$, $\sin x$, $\cos x$
        kombinasyonlari olarak ortaya cikar.
  \item Sag tarafta $x^n, e^{ax}, \sin bx, \cos bx$ gibi fonksiyonlar varsa
        ozel cozum tahminleri dogrudan bu form uzerinden yapilir.
\end{itemize}

\subsection{Grafikler ve Temel Sekiller}

\begin{itemize}[leftmargin=*]
  \item Polinomlar genel olarak \emph{puruzsuz} (her yerde turevli) eg riler verir.
  \item $e^x$ her zaman pozitiftir ve $x$ arttikca hizla artar.
  \item $\ln x$ yalnizca $x>0$ icin tanimlidir ve yavas artan bir fonksiyondur.
  \item $\sin x$ ve $\cos x$ periyod ik ve genligi sinirli fonksiyonlardir.
\end{itemize}

Bu sekilleri kafanda canlandirabilmek, c ozumlerin mantikli olup olmadigini
kontrol etmek icin faydalidir.

\subsection{Limit ve Sureklilik ile Baglanti}

\begin{itemize}[leftmargin=*]
  \item Polinomlar her yerde surekli ve turevlenebilirdir.
  \item $e^x$, $\sin x$, $\cos x$ gibi temel fonksiyonlar da ayni ozellige sahiptir.
  \item $\ln x$ yalnizca $x>0$ icin tanimli oldugundan, $x \to 0^+$ davranisi
        (limit) onem kazanir.
\end{itemize}

\paragraph{Ders icin mesaj.}
Bu bolumdeki kavramlar, diferansiyel denklemlerin \emph{on yuzu} gibidir:
karakteristik denklemler, temel cozum tipleri ve cozumun buyuk $x$'lerdeki
veya belirli noktalardaki davranisi icin bu fonksiyonlara hakim olmak gerekir.

\section*{Polinom Ornekleri ve Alistirmalar}

Asagidaki kisa ornekler, polinomlarla hizli islem yapma ve karakteristik
polinom fikrini pekistirmek icin hazirlanmistir.

\begin{enumerate}[leftmargin=*]
  \item $P(x) = 3x^2 - 5x + 2$ polinomunun derecesini, bas katsayisini ve sabit terimini yaz.\\
      extbf{Sonuc:} Derece $2$, bas katsayi $3$, sabit terim $2$.

  \item $P(x) = 2x^3 - x + 1$ icin $P(0)$, $P(1)$ ve $P(-1)$ degerlerini hesapla.\\
      extbf{Sonuc:} $P(0)=1$, $P(1)=2$, $P(-1)=3$.

  \item $P(x) = x^2 - 4x + 3$ polinomunun koklerini bul ve polinomu carp anlara ayir.\\
      extbf{Sonuc:} Kokler $x=1$ ve $x=3$; $P(x) = (x-1)(x-3)$.

  \item $P(x) = x^2 + 5x + 6$ icin kokleri bul; benzer sekilde carp anlara ayir.\\
      extbf{Sonuc:} Kokler $x=2$ ve $x=3$; $P(x) = (x-2)(x-3)$.

  \item $(x+2)(x-3)$ carpimini acip bir polinom olarak yaz; derecesini belirt.\\
      extbf{Sonuc:} $(x+2)(x-3) = x^2-x-6$, derece $2$.

  \item $(2x-1)(x^2+3)$ carpimini ac ve elde ettigin polinomun katsayilarini yaz.\\
      extbf{Sonuc:} $2x^3 - x^2 + 6x - 3$; katsayilar $2, -1, 6, -3$.

  \item Uzun bolme (veya sentetik bolme) kullanarak $x^3 - 1$ polinomunu $(x-1)$'e bol.\\
      extbf{Sonuc:} $x^3-1 = (x-1)(x^2+x+1)$.

  \item $y'' - 5y' + 6y = 0$ icin karakteristik denklemi yaz ve koklerini bul.\\
      extbf{Sonuc:} Karakteristik denklem $\lambda^2 - 5\lambda + 6 = 0$; kokler $\lambda=2$ ve $\lambda=3$.

  \item Bir onceki maddede buldugun koklere gore genel cozumun $y(x)$ seklini tahmin et.\\
      extbf{Sonuc:} $y(x) = C_1 e^{2x} + C_2 e^{3x}$.

  \item $P(x) = x^4 - 1$ polinomunu carp anlara ayir: once ikinci derecedenlere, sonra
    istersen gercek koklere kadar devam et.\\
      extbf{Sonuc:} $x^4-1 = (x^2-1)(x^2+1) = (x-1)(x+1)(x^2+1)$; gercek kokler $x=\pm 1$.
\end{enumerate}

% % Limit ve Sureklilik

\section{Limit ve Sureklilik}

Bu bolum, turev ve diferansiyel denklemler oncesi analiz altyapisini hizlica
hatirlatmak icin ozet niteligindedir.

\subsection{Limit Fikri (Gayriresmi)}

Bir $f(x)$ fonksiyonunun $x \to a$ iken limiti $L$ ise, asagidaki orneklerle
bu fikri pekistirebilirsin:

\begin{itemize}[leftmargin=*]
  \item Basit cebirsel fonksiyonlar icin limit genellikle dogrudan yerine koyma ile bulunur.
  \item Paydada $0$ durumu varsa sadeleştirme veya daha ince analiz gerekir.
\end{itemize}

Ardindan kisa alistirmalar:

\begin{enumerate}[leftmargin=*]
  \item $x\to 2$ icin $f(x) = 3x+1$ fonksiyonunun limitini hesapla.\\
      extbf{Sonuc:} $\lim\limits_{x\to 2} (3x+1) = 3\cdot 2+1 = 7$.
  \item $x\to 1$ icin $f(x) = x^2 - 1$ fonksiyonunun limitini bul ve fonksiyonun
    $x=1$'de surekli olup olmadigini yorumla.\\
      extbf{Sonuc:} $\lim\limits_{x\to 1} (x^2-1) = 0$ ve $f(1)=0$ oldugu icin fonksiyon $x=1$'de sureklidir.
  \item $x\to 0$ icin $f(x) = \dfrac{\sin x}{x}$ fonksiyonunun limitini hesapla.\\
      extbf{Sonuc:} $\displaystyle \lim_{x\to 0} \dfrac{\sin x}{x} = 1$.
  \item $x\to 0$ icin $f(x) = \dfrac{1-\cos x}{x^2}$ fonksiyonunun limitini bul.\\
      extbf{Sonuc:} $\displaystyle \lim_{x\to 0} \dfrac{1-\cos x}{x^2} = \dfrac{1}{2}$.
  \item $x\to \infty$ icin $f(x) = \dfrac{2x^2+1}{x^2+3}$ fonksiyonunun limitini hesapla.\\
      extbf{Sonuc:} En baskin terimler oranindan $\displaystyle \lim_{x\to \infty} \dfrac{2x^2+1}{x^2+3} = 2$.
  \item $x\to \infty$ icin $f(x) = \dfrac{3x-5}{2x+7}$ fonksiyonunun limitini bul.\\
      extbf{Sonuc:} $\displaystyle \lim_{x\to \infty} \dfrac{3x-5}{2x+7} = \dfrac{3}{2}$.
  \item $x\to 0$ icin $f(x) = |x|$ fonksiyonunun limitini incele; soldan ve sagdan limitleri
    karsilastir.\\
      extbf{Sonuc:} $\lim\limits_{x\to 0^-} |x| = 0$, $\lim\limits_{x\to 0^+} |x| = 0$ ve $f(0)=0$; limit vardir ve fonksiyon $0$'da sureklidir.
  \item $f(x) = \begin{cases}
  x^2, & x<1, \\
  2x-1, & x\ge 1
    \end{cases}$ icin $x\to 1$ limitini ve $x=1$'de surekliligini incele.\\
      extbf{Sonuc:} Soldan limit $1^2=1$, sagdan limit $2\cdot 1-1=1$ ve $f(1)=1$;
    dolayisiyla $x=1$'de limit vardir ve fonksiyon sureklidir.
  \item $f(x) = \dfrac{x^2-1}{x-1}$ fonksiyonu icin $x\to 1$ limitini hesapla ve fonksiyonun
    $x=1$'de tanimli olup olmadigini yorumla.\\
      extbf{Sonuc:} $\dfrac{x^2-1}{x-1} = x+1$ (\(x\neq 1\) icin), dolayisiyla $\lim\limits_{x\to 1} f(x) = 2$;
    fakat $x=1$ noktasi tanim kumesinde degildir, burada kaldirilabilir bir sureksizlik vardir.
  \item $f(x) = \begin{cases}
  x+1, & x\ne 0, \\
  0, & x=0
    \end{cases}$ fonksiyonu icin $x\to 0$ limitini ve $x=0$'de surekliligini incele.\\
      extbf{Sonuc:} $x\to 0$ icin $x+1\to 1$, yani $\lim\limits_{x\to 0} f(x)=1$;
    ancak $f(0)=0$ oldugu icin fonksiyon $0$'da sureksizdir.
\end{enumerate}

\paragraph{Ders icin mesaj.}
Turev tanimi limit uzerinden yapildigi icin, bu kurallari refleks haline getirmek
hesaplamalari hizlandirir.

\paragraph{Mini ornekler.}
\begin{align*}
  \lim_{x \to 2} (3x^2 - x)
  &= 3\cdot 2^2 - 2 = 12 - 2 = 10, \\
  \lim_{x \to 1} \frac{x^2 - 1}{x-1}
  &= \lim_{x \to 1} (x+1) = 2.
\end{align*}

\subsection{Tek Tarafli ve Sonsuzda Limit}

\begin{itemize}[leftmargin=*]
  \item Sag limit: $x \to a^+$, $x$ degerleri $a$'ya \emph{sagdan} yaklasir.
  \item Sol limit: $x \to a^-$, $x$ degerleri $a$'ya \emph{soldan} yaklasir.
  \item Sag ve sol limit esitse ortak limit vardir.
\end{itemize}

Sonsuzda limitler:
\[
\lim_{x \to \infty} f(x),\qquad \lim_{x \to -\infty} f(x)
\]
fonksiyonun uc noktalardaki davranisini verir.

\paragraph{Ornek.}
\[
\lim_{x \to \infty} \frac{1}{x} = 0.
\]

\paragraph{Mini ornek.}
\[
  \lim_{x \to \infty} \frac{2x^2 + 1}{x^2 - 3}
  = \lim_{x \to \infty} \frac{2 + 1/x^2}{1 - 3/x^2} = 2.
\]

Bu tur dusunceler, ODE cozumlerinin uzun vadede neye yaklastigini (kararlilik,
vanis etme, salinim devam ediyor mu vb.) yorumlarken ise yarar.

\subsection{Sureklilik}

Bir $f$ fonksiyonu $x=a$ noktasinda \emph{s\"urekli} ise:
\begin{enumerate}[leftmargin=*]
  \item $f(a)$ tanimlidir,
  \item $\displaystyle \lim_{x \to a} f(x)$ vardir,
  \item $\displaystyle \lim_{x \to a} f(x) = f(a)$ saglanir.
\end{enumerate}

S\"ureksizlik turleri:
\begin{itemize}[leftmargin=*]
  \item atlamali s\"ureksizlik,
  \item sonsuz s\"ureksizlik,
  \item tanimsiz nokta vb.
\end{itemize}

\paragraph{Mini ornekler.}
\begin{itemize}[leftmargin=*]
  \item $f(x) = \dfrac{1}{x}$ icin $x=0$'da sonsuz s"ureksizlik (dikey asimptot) vardir.
  \item $g(x) = \begin{cases}
          1, & x<0, \\
          2, & x\ge 0
        \end{cases}$ icin $x=0$'da atlamali s"ureksizlik vardir.
\end{itemize}

Polinomlar, ustel fonksiyonlar ve $\sin, \cos$ gibi temel fonksiyonlar
her yerde s\"ureklidir.

\subsection{Turevlenebilirlik ve Sureklilik}

Bir nokta icin:
\begin{itemize}[leftmargin=*]
  \item $f$ o noktada \emph{turevlenebilir} ise, o noktada mutlaka s\"ureklidir.
  \item Tersi her zaman dogru degildir (ornegin $|x|$ fonksiyonunda $x=0$).
\end{itemize}

Turev tanimi:
\[
f'(a) = \lim_{h \to 0} \frac{f(a+h) - f(a)}{h}.
\]

Bu tanimin temelinde hep limit kurallari yatar; pratikte formulleri hazir
kullansan bile, arkadaki teori buraya dayanir.

\subsection{Diferansiyel Denklemlerle Baglanti}

\begin{itemize}[leftmargin=*]
  \item Diferansiyel denklemlerde cozumler genellikle turev ve s\"ureklilik
        varsayimlari uzerine kurulur.
  \item Cozum fonksiyonunun tanim araligini belirlerken limit ve s\"ureklilik
        davranisini dikkate almalisin (ozellikle $\ln x$, kok iceren fonksiyonlar vb.).
\end{itemize}

Bu bolumdeki fikirler, sinavda uzun ispatlar seklinde sorulmasa bile, hangi
fonksiyonlarla rahat calisabilecegini ve nerede dikkatli olman gerektigini
belirlemede sana temel bir sezgi kazandirir.

\section*{Limit ve Sureklilik Ornekleri}

Kisa alistirmalarla limit ve sureklilik kavramlarini pekistirmek icin asagidaki
ornekleri cozebilirsin.

\begin{enumerate}[leftmargin=*]
  \item $\displaystyle \lim_{x \to 2} (x^2 + 3x - 1)$ degerini hesapla.\\
    	extbf{Sonuc:} $x=2$ yazarsan $2^2 + 3\cdot 2 - 1 = 4+6-1 = 9$.
  \item $\displaystyle \lim_{x \to 1} \frac{x^2 - 1}{x-1}$ limitini sadeleştirerek bul.\\
    	extbf{Sonuc:} $\dfrac{x^2-1}{x-1} = x+1$ (\(x\neq 1\) icin), bu yuzden $\lim_{x\to 1} = 2$.
  \item $\displaystyle \lim_{x \to 0} \frac{\sin x}{x}$ icin teorik sonucu hatirla
    (seriler veya grafik uzerinden) ve sonuclandir.\\
      	extbf{Sonuc:} $\displaystyle \lim_{x\to 0} \frac{\sin x}{x} = 1$.
  \item $\displaystyle \lim_{x \to \infty} \frac{3x^2 - x + 1}{x^2 + 4}$ limitini
    pay ve paydayi $x^2$'ye bolerek hesapla.\\
      	extbf{Sonuc:} Baskin terimlerden $\displaystyle \lim_{x\to\infty} \frac{3x^2 - x + 1}{x^2 + 4} = 3$.
  \item $\displaystyle \lim_{x \to 0^+} \ln x$ ve $\displaystyle \lim_{x \to \infty} \ln x$
    limitlerini yorumla.\\
      	extbf{Sonuc:} $x\to 0^+$ icin $\ln x \to -\infty$, $x\to \infty$ icin $\ln x \to \infty$.
  \item
    $f(x) = \begin{cases}
      x^2, & x<1, \\
      2x-1, & x\ge 1
    \end{cases}$ icin $x=1$ noktasinda sag ve sol limitleri, ayrica $f(1)$ degerini hesapla;
    fonksiyonun bu noktada surekli olup olmadigini belirt.\\
      	extbf{Sonuc:} Sol limit $1^2=1$, sag limit $2\cdot 1-1=1$, $f(1)=1$; bu nedenle fonksiyon $x=1$'de sureklidir.
  \item
    $g(x) = \dfrac{1}{x-2}$ fonksiyonu icin $x\to 2^-$ ve $x\to 2^+$ limitlerini incele;
    bu noktada hangi turde bir sureksizlik vardir?\\
      	extbf{Sonuc:} $x\to 2^-$ icin $g(x)\to -\infty$, $x\to 2^+$ icin $g(x)\to +\infty$;
        bu noktada sonsuz (asil) bir sureksizlik vardir.
  \item
  $h(x) = \begin{cases}
      x, & x\ne 0, \\
      1, & x=0
    \end{cases}$ fonksiyonu icin $x=0$'da limit ve fonksiyon degerini karsilastir;
    surekli mi, degil mi?\\
      	extbf{Sonuc:} $\lim_{x\to 0} h(x) = 0$, fakat $h(0)=1$; limit ve deger farkli oldugu icin $x=0$'da sureksizdir.
  \item $\displaystyle \lim_{x \to -\infty} e^x$ ve $\displaystyle \lim_{x \to \infty} e^{-x}$
    limitlerini hesapla; grafiksel yorum yap.\\
      	extbf{Sonuc:} $x\to -\infty$ icin $e^x\to 0$, $x\to \infty$ icin $e^{-x}\to 0$; her iki durumda da eksene yakinlasan ama kesmeyen bir kuyruk vardir.
  \item Bir fonksiyonun bir noktada turevlenebilir olmasi icin hangi sartin once
    saglanmasi gerektigini (sureklilik ile iliskisi) kisa bir cumleyle acikla.\\
      	extbf{Sonuc:} Bir noktada turevlenebilir olmak icin once o noktada surekli olmak gerekir; turevlenebilirlik surekliligi garanti eder ama tersi her zaman dogru degildir.
\end{enumerate}

% % Katli Integraller Modul Dosyasi

\section{Katli Integraller}

Bu bolum, iki ve uc katli integralleri \emph{recete gibi} hatirlamak icin kisa bir ozet sunar.

\subsection{Iki Katli Integral: Temel Fikir}

Suresiz ve uygun sekilde duzenli bir \(f(x,y)\) fonksiyonu icin, dikdortgensel bir bolge uzerinde
\(\iint\) su sekilde tanimlanir. \(R = [a,b] \times [c,d]\) olsun. O zaman
\begin{align*}
  \iint_R f(x,y)\,\mathrm{d}A
  &= \int_a^b \int_c^d f(x,y)\,\mathrm{d}y\,\mathrm{d}x \\
  &= \int_c^d \int_a^b f(x,y)\,\mathrm{d}x\,\mathrm{d}y.
\end{align*}

Genelde, hangi siralama kolay geliyorsa onu secersin; ama \emph{sinirlarin dogru yazilmasi} kritik nokta.

\paragraph{Hesaplanmis ornek (dikdortgen bolge).}
\[
  \int_0^1 \int_0^2 (x + y)\,\mathrm{d}y\,\mathrm{d}x.
\]
Once ic integrali (\(y\) ye gore) al:
\begin{align*}
  \int_0^2 (x+y)\,\mathrm{d}y
  &= \left[ xy + \tfrac{1}{2}y^2 \right]_{y=0}^{y=2}
   = 2x + 2.
\end{align*}
Sonra dis integrali (\(x\) e gore) al:
\begin{align*}
  \int_0^1 (2x+2)\,\mathrm{d}x
  &= \left[ x^2 + 2x \right]_0^1 = 1 + 2 = 3.
\end{align*}
Yani iki katli integralin sonucu \(3\)'tur.

\subsection{Dikdortgen Olmayan Bolgeler}

Pratikte siklikla asagidaki tip bolgelerle karsilasirsin:
\begin{itemize}
  \item \(D = \{(x,y) : a \le x \le b,\ g_1(x) \le y \le g_2(x)\}\)
  \item \(D = \{(x,y) : c \le y \le d,\ h_1(y) \le x \le h_2(y)\}\)
\end{itemize}

Ilk tip icin
\[
  \iint_D f(x,y)\,\mathrm{d}A = \int_a^b \left( \int_{g_1(x)}^{g_2(x)} f(x,y)\,\mathrm{d}y \right)\mathrm{d}x,
\]
ikinci tip icin benzer sekilde once \(x\), sonra \(y\) uzerinden integral alirsin.

\paragraph{Pratik recete.}
\begin{enumerate}[label=\arabic*)]
  \item Bolgeyi mutlaka kabaca da olsa ciz.
  \item \(x\) veya \(y\)'ye gore bir \emph{kesit} al: alt ve ust sinir fonksiyonlarini netlestir.
  \item Integral siralamasini bu kesite gore yaz (\(y\) icte, \(x\) diste ya da tam tersi).
\end{enumerate}

\subsection{Koordinat Donusumleri (Ozet)}

Bazi katli integraller, uygun bir koordinat donusumu ile cok daha kolay hale gelir.
En temel ornek \textbf{polar (kutupsal) koordinatlar}dir.

\subsubsection*{Polar koordinatlar}

\[
  x = r \cos\theta, \qquad y = r \sin\theta.
\]

Bu donusumde alan elemani
\[
  \mathrm{d}A = r\,\mathrm{d}r\,\mathrm{d}\theta
\]
seklinde gelir; yani integrale ekstra bir \(r\) carpani eklenir.
Genel form:
\[
  \iint_D f(x,y)\,\mathrm{d}A
  = \int_{\theta_1}^{\theta_2} \int_{r_1(\theta)}^{r_2(\theta)}
      f(r\cos\theta, r\sin\theta)\, r\,\mathrm{d}r\,\mathrm{d}\theta.
\]

Daire, disk, halka gibi dairesel simetrili bolgelerde polar koordinat secimi neredeyse her zaman isin kolaylastirir.

\paragraph{Hesaplanmis ornek (disk uzerinde integral).}

\(D\), merkezde yaricapi 1 olan birim disk olsun: \(x^2 + y^2 \le 1\). \(f(x,y)=x^2 + y^2\)
icin
\[
  \iint_D (x^2 + y^2)\,\mathrm{d}A
\]
degerini hesaplayalim.

Polar koordinatlara gecersek \(x^2 + y^2 = r^2\) ve \(\mathrm{d}A = r\,\mathrm{d}r\,\mathrm{d}\theta\) oldugundan,
bolge \(0 \le r \le 1\), \(0 \le \theta \le 2\pi\) ile tarif edilir:
\begin{align*}
  \iint_D (x^2 + y^2)\,\mathrm{d}A
  &= \int_0^{2\pi} \int_0^1 r^2 \cdot r\,\mathrm{d}r\,\mathrm{d}\theta \\
  &= \int_0^{2\pi} \int_0^1 r^3\,\mathrm{d}r\,\mathrm{d}\theta \\
  &= \int_0^{2\pi} \left[ \tfrac{1}{4} r^4 \right]_0^1 \mathrm{d}\theta \\
  &= \int_0^{2\pi} \tfrac{1}{4}\,\mathrm{d}\theta \\
  &= \tfrac{1}{4} \cdot 2\pi = \frac{\pi}{2}.
\end{align*}

\subsection{Uc Katli Integraller}

Uc degiskenli bir \(f(x,y,z)\) fonksiyonu icin
\[
  \iiint_E f(x,y,z)\,\mathrm{d}V
\]
ifadesi, hacim uzerinde integral anlamina gelir. Dikdortgensel bir bolge icin
\[
  E = [a,b] \times [c,d] \times [e,f]
\]
oldugunda, integral ic ice uc tekli integral seklinde yazilir:
\[
  \iiint_E f(x,y,z)\,\mathrm{d}V
  = \int_a^b \int_c^d \int_e^f f(x,y,z)\,\mathrm{d}z\,\mathrm{d}y\,\mathrm{d}x,
\]
veya degisken siralamasi ihtiyaca gore degistirilebilir.

\subsection{Diferansiyel Denklemlerle Baglanti}

Baslangic diferansiyel denklemleri dersinde katli integraller cok sik sorulmayabilir;
ancak daha ileri konularda ve fiziksel modellere gecince:
\begin{itemize}
  \item Hacim/alan hesaplari,
  \item Yogunluk fonksiyonu ile \emph{toplam kutle},
  \item Ortalama deger hesaplari
\end{itemize}
hep katli integral fikrine dayanir.

Ote yandan, PDE (kismen diferansiyel denklemler) tarafinda sinir deger problemleri,
enerji/isi korunumu gibi konulara girdiginde, bu sayfaya geri donup
\emph{"bolge uzerinde integral nasil yaziliyordu"} diye bakmak isini kolaylastirir.

\section*{Katli Integral Ornekleri ve Alistirmalar}

Asagidaki ornekler, iki katli ve uc katli integrallerin kurulum ve hesaplamasini
pratik etmen icin hazirlanmistir.

\begin{enumerate}[leftmargin=*]
  \item $\displaystyle \int_0^1 \int_0^1 (x + 2y)\,\mathrm{d}y\,\mathrm{d}x$ integralini hesapla.\\
        	extbf{Sonuc:} Ic integral $\left[xy + y^2\right]_0^1 = x+1$;
        dis integral $\int_0^1 (x+1)\,dx = \left[\tfrac{x^2}{2} + x\right]_0^1 = \tfrac{3}{2}$.

  \item $\displaystyle \int_0^2 \int_0^1 (3x - y)\,\mathrm{d}x\,\mathrm{d}y$ icin once ic, sonra dis
    integrali alarak sonucu bul.\\
        	extbf{Sonuc:} Ic integral $\left[\tfrac{3}{2}x^2 - yx\right]_0^2 = 6-2y$;
        dis integral $\int_0^1 (6-2y)\,dy = [6y - y^2]_0^1 = 5$.

  \item $\displaystyle \int_0^1 \int_0^2 (x^2 + y)\,\mathrm{d}y\,\mathrm{d}x$ integralini hesaplarken,
    ister $y$'ye gore ister $x$'e gore once integre etmeyi dene (sonuc ayni olacak).\\
        	extbf{Sonuc:} $\displaystyle \int_0^1 \left[x^2 y + \tfrac{y^2}{2}\right]_0^2 dx = \int_0^1 (2x^2 + 2) dx = \left[\tfrac{2}{3}x^3 + 2x\right]_0^1 = \tfrac{8}{3}$.

  \item $D = [0,1] \times [0,3]$ icin $\displaystyle \iint_D (2x + y)\,\mathrm{d}A$ integralini
    acik integral seklinde yaz ve hesapla.\\
        	extbf{Sonuc:} $\displaystyle \int_0^1 \int_0^3 (2x+y)\,dy\,dx = \int_0^1 (6x+\tfrac{9}{2}) dx
        = \left[3x^2 + \tfrac{9}{2}x\right]_0^1 = \tfrac{15}{2}$.

  \item $D = \{(x,y) : 0 \le x \le 1,\ x \le y \le 1\}$ bolgesi icin
    $\displaystyle \iint_D y\,\mathrm{d}A$ integralini once $y$ sonra $x$ uzerinden yaz.\\
            extbf{Sonuc:} $\displaystyle \int_0^1 \int_x^1 y\,dy\,dx = \int_0^1 \left[\tfrac{y^2}{2}\right]_x^1 dx
        = \int_0^1 \left(\tfrac{1}{2} - \tfrac{x^2}{2}\right) dx = \left[\tfrac{x}{2} - \tfrac{x^3}{6}\right]_0^1 = \tfrac{1}{3}$.

  \item Yukaridaki $D$ bolgesi icin ayni integrali bu kez once $x$, sonra $y$ uzerinden
    yazmayi dene (limitleri degisken fonksiyonlar seklinde ifade et).\\
          extbf{Sonuc:} $D$ icin $0 \le y \le 1$, $0 \le x \le y$; integral $\displaystyle \int_0^1 \int_0^y y\,dx\,dy
  = \int_0^1 y^2 \,dy = \left[\tfrac{y^3}{3}\right]_0^1 = \tfrac{1}{3}$ (ayni sonuc).

  \item Birim kare $[0,1]\times[0,1]$ uzerinde $f(x,y)=x^2y$ icin ortalama deger
    $\displaystyle f_{ort} = \frac{1}{\text{alan}} \iint f\,\mathrm{d}A$ formuluyle hesapla.\\
          extbf{Sonuc:} Alan 1 oldugu icin $f_{ort} = \displaystyle \int_0^1\int_0^1 x^2 y\,dy\,dx
  = \int_0^1 x^2 \left[\tfrac{y^2}{2}\right]_0^1 dx = \int_0^1 \tfrac{x^2}{2} \,dx = \tfrac{1}{6}$.

  \item $E = [0,1]\times[0,1]\times[0,1]$ icin
    $\displaystyle \iiint_E (x + y + z)\,\mathrm{d}V$ hacim integralini hesapla.\\
          extbf{Sonuc:} Ayrilabilir: $\displaystyle \iiint_E (x+y+z)\,dV = \int_0^1 x\,dx + \int_0^1 y\,dy + \int_0^1 z\,dz
  = 3 \cdot \tfrac{1}{2} = \tfrac{3}{2}$.

  \item Polar koordinatlara gecerek, yaricapi 2 olan diskte
    $\displaystyle \iint_D x^2 \,\mathrm{d}A$ integralini kur (hesaplamayi istersen tam
    yap, istersen kurulmus sekilde birak).\\
          extbf{Sonuc:} $x = r\cos\theta$, $\mathrm{d}A = r\,dr\,d\theta$; integrand $x^2 = r^2 \cos^2\theta$;
  integral $\displaystyle \int_0^{2\pi} \int_0^2 r^3 \cos^2\theta\,dr\,d\theta$ seklinde yazilir.

  \item Yine polar koordinatlarla, $\displaystyle \iint_D 1\,\mathrm{d}A$ integralinin aslinda
    bolgenin alanini verdigini hatirla ve yaricapi $R$ olan bir diskin alanini bu sekilde bul.\\
          extbf{Sonuc:} $\displaystyle \int_0^{2\pi} \int_0^R 1\cdot r\,dr\,d\theta = \int_0^{2\pi} \left[\tfrac{r^2}{2}\right]_0^R d\theta
  = \int_0^{2\pi} \tfrac{R^2}{2} \,d\theta = \pi R^2$.
\end{enumerate}

% % Diferansiyel Denklemler Ozet

\section{Diferansiyel Denklemler Ozet}

Bu bolum, sinavda en cok isine yarayacak temel ODE turleri icin kisa bir
\emph{recete} sunar.

\subsection{Temel Tanimlar}

\begin{itemize}[leftmargin=*]
  \item \textbf{Mertebe (order):} Denklemde gecen en yuksek turevin derecesi.
    Ornek: $y'' + 3y' - 2y = 0$ \,$\Rightarrow$\, 2. mertebe.
  \item \textbf{Lineer ODE:} $y, y', y'', \dots$ terimleri 1. dereceden, kendi
    aralarinda carpim yok. Ornek: $y' + p(x)y = q(x)$ lineer; $y^2 y' = x$ lineer degil.
  \item \textbf{Genel cozum:} Icinde sabitler ($C, C_1, C_2, \dots$) bulunan cozum ailesi.
  \item \textbf{Ozel cozum:} Baslangic/kenar kosullari yerlestirince elde edilen belirli cozum.
\end{itemize}

\subsection{Ayrilabilir Denklemler}

\textbf{Form:}
\[
\frac{dy}{dx} = f(x) g(y)
\]
veya buna denk bir yazilis.

\textbf{Recete:}
\begin{enumerate}[leftmargin=*]
  \item Tum $y$ terimlerini bir tarafa, $x$ terimlerini diger tarafa topla:
    \[
    \frac{1}{g(y)} \, dy = f(x) \, dx.
    \]
  \item Her iki tarafi da integre et:
    \[
    \int \frac{1}{g(y)} \, dy = \int f(x) \, dx + C.
    \]
  \item Mumkunse $y$'yi yalniz birak.
\end{enumerate}

\paragraph{Mini ornek.}
\[
\frac{dy}{dx} = x y
\]
Ayir:
\[
\frac{1}{y} \, dy = x \, dx.
\]
Integralle:
\[
\ln|y| = \frac{x^2}{2} + C \quad \Rightarrow \quad y = C e^{x^2/2}.
\]

Bu tip denklemler, temel seviye fizik ve uygulama sorularinda sik karsina cikar.

\subsection{Birinci Mertebe Lineer Denklemler}

\textbf{Genel form:}
\[
y' + p(x) y = q(x).
\]

\textbf{Recete (entegrasyon faktoru):}
\begin{enumerate}[leftmargin=*]
  \item $p(x)$'i belirle.
  \item Entegrasyon faktorunu hesapla:
    \[
    \mu(x) = e^{\int p(x) \, dx}.
    \]
  \item Denklemin her iki yanini $\mu(x)$ ile carp:
    \[
    \mu(x) y' + \mu(x) p(x) y = \mu(x) q(x).
    \]
  \item Sol taraf artik bir turevdir:
    \[
    (\mu(x) y)' = \mu(x) q(x).
    \]
  \item Her iki tarafi integra et:
    \[
    \mu(x) y = \int \mu(x) q(x) \, dx + C.
    \]
  \item Son olarak $y$'yi yalniz birak:
    \[
    y(x) = \frac{1}{\mu(x)}\left( \int \mu(x) q(x) \, dx + C \right).
    \]
\end{enumerate}

\paragraph{Not.} Ana dosyanin turev-integral bolumunde bu tipe ait detayli bir ornek var.

\subsection{Sabit Katsayili 2. Mertebe Lineer ODE'ler}

\textbf{Genel form:}
\[
ay'' + by' + cy = 0, \quad a \ne 0.
\]

\textbf{Adim 1: Karakteristik denklem.}
\[
a\lambda^2 + b\lambda + c = 0.
\]

Cozumler kok turune gore ayrilir.

\subsubsection{Iki Gercek Ayri k Kok (iki farkli gercek kok)}

Karakteristik denklemden iki farkli gercek kok elde edildigini varsayalim; bunlari
$\lambda_1$ ve $\lambda_2$ ile gosterirsek genel cozum
\[
y(x) = C_1 e^{\lambda_1 x} + C_2 e^{\lambda_2 x}
\]
seklindedir.

\subsubsection{Cakisik Gercek Kok (cakisik gercek kok)}

Eger karakteristik denklemden tek bir gercek kok (cakisik kok) cikarsa, bu koku
$\lambda$ ile gosteririz ve genel cozum
\[
y(x) = (C_1 + C_2 x) e^{\lambda x}
\]
seklindedir.

\subsubsection{Karmasik Esl enik Kokler (alfa artı-eksi i beta turu karmasik kokler)}

Karakteristik denklemden karmasik eslenik kokler elde edersek, bunlari
$\lambda_{1,2} = \alpha \pm i\beta$ seklinde yazabiliriz. Bu durumda genel cozum
\[
y(x) = e^{\alpha x} (C_1 \cos(\beta x) + C_2 \sin(\beta x))
\]
bi\c cimindedir.

\paragraph{Ozet.}
\begin{itemize}[leftmargin=*]
  \item $\Delta = b^2 - 4ac > 0$ ise: iki gercek ayri k kok.
  \item $\Delta = 0$ ise: cakisik kok.
  \item $\Delta < 0$ ise: karmasik kokler.
\end{itemize}

\subsection{Zorlanmis (Sag Tarafli) Sabit Katsayili Lineer ODE}

\textbf{Form:}
\[
ay'' + by' + cy = g(x).
\]

\textbf{Genel strateji:}
\begin{enumerate}[leftmargin=*]
  \item Once homojen kismi coz: $ay'' + by' + cy = 0$ \,$\Rightarrow$\, $y_h$.
  \item Sonra, $g(x)$'in turune gore bir \emph{ozel cozum} $y_p$ tahmin et:
    \begin{itemize}[leftmargin=*]
      \item $g(x)$ polinom ise: polinom formunda dene.
      \item $g(x) = e^{kx}$ ise: $Ae^{kx}$ dene.
      \item $g(x) = \sin bx$ veya $\cos bx$ ise: $A\cos bx + B\sin bx$ dene.
    \end{itemize}
  \item Toplam cozum: $y = y_h + y_p$.
\end{enumerate}

Detayli hesap, ders seviyene gore degisebilir; fakat bu \emph{iskelet} hep aynidir.

\subsection{Fiziksel Yorumlar (Cok Kisa)}

\begin{itemize}[leftmargin=*]
  \item \textbf{Harmonik osilator:} $y'' + \omega^2 y = 0$  $\Rightarrow$ sin\"us-kosin\"us tipinde salinim cozumleri.
  \item \textbf{Sonumlu salinim:} $y'' + 2\gamma y' + \omega^2 y = 0$  
    icin koklerin dogasi (gercek/cakisik/karmasik) sistemin ne kadar hizli sonumlend igini belirler.
\end{itemize}

Bu tur yorumlar, elde ettigin cozumun formunun fiziksel olarak mantikli olup
olmadigini kontrol etmek icin iyi bir sezgi kaynagi saglar.

\section*{Diferansiyel Denklem Ornekleri ve Alistirmalar}

Asagidaki kisa ornekler, farkli ODE turleri icin receteleri uygulama alistirmasi
olarak dusunulebilir. Cogu icin sadece genel cozumu bulman yeterlidir.

\begin{enumerate}[leftmargin=*]
  \item $\displaystyle \frac{dy}{dx} = 3y$ icin ayrilabilirlik recetesini kullanarak genel cozumu bul.
  \item $\displaystyle \frac{dy}{dx} = x y$ denklemi icin ayirma adimlarini yaz ve genel cozumu elde et.
  \item $\displaystyle y' + y = 0$ denklemini birinci mertebe lineer receteyle coz.
  \item $\displaystyle y' - 2y = e^{3x}$ denklemi icin integrasyon faktorunu yaz ve genel cozumu bul.
    \textbf{Cozum:} Entegrasyon faktoru $\mu(x) = e^{\int -2 dx} = e^{-2x}$.
    Denklemi carparsak: $(e^{-2x}y)' = e^{-2x}e^{3x} = e^x$.
    Integral alirsak: $e^{-2x}y = \int e^x dx = e^x + C$.
    Genel cozum: $y(x) = e^{3x} + Ce^{2x}$.
  \item $y'' - 3y' + 2y = 0$ icin karakteristik denklemi yaz, kokleri bul ve genel cozumu belirt.
  \item $y'' + y = 0$ denklemi icin cozumun sin ve cos cinsinden genel formunu yaz.
  \item $y'' + 4y = 0$ denkleminin karakteristik koklerini ve buna karsilik gelen salinim
    frekansini (omega) yorumla.
  \item $y'' - y = e^x$ gibi zorlanmis bir denklem icin (homojen + ozel cozum) genel cozum
    stratejisini kisa maddeler halinde ozetle (detayli hesap zorunlu degil).
  \item Baslangic deger problemi: $y' = 2y$, $y(0) = 3$ icin once genel cozumu bul, sonra baslangic
    kosulunu kullanarak sabiti belirle.
  \item $y'' + 2y' + y = 0$ denklemi icin $\Delta = b^2 - 4ac$ degerini hesapla ve koklerin hangi
    tipte oldugunu (ayrik, cakisik, karmasik) siniflandir; buna gore genel cozumu yaz.
\end{enumerate}


\end{document}
