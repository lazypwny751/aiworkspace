% Diferansiyel Denklemler Ozet

\section{Diferansiyel Denklemler Ozet}

Bu bolum, sinavda en cok isine yarayacak temel ODE turleri icin kisa bir
\emph{recete} sunar.

\subsection{Temel Tanimlar}

\begin{itemize}[leftmargin=*]
  \item \textbf{Mertebe (order):} Denklemde gecen en yuksek turevin derecesi.
    Ornek: $y'' + 3y' - 2y = 0$ \,$\Rightarrow$\, 2. mertebe.
  \item \textbf{Lineer ODE:} $y, y', y'', \dots$ terimleri 1. dereceden, kendi
    aralarinda carpim yok. Ornek: $y' + p(x)y = q(x)$ lineer; $y^2 y' = x$ lineer degil.
  \item \textbf{Genel cozum:} Icinde sabitler ($C, C_1, C_2, \dots$) bulunan cozum ailesi.
  \item \textbf{Ozel cozum:} Baslangic/kenar kosullari yerlestirince elde edilen belirli cozum.
\end{itemize}

\subsection{Ayrilabilir Denklemler}

\textbf{Form:}
\[
\frac{dy}{dx} = f(x) g(y)
\]
veya buna denk bir yazilis.

\textbf{Recete:}
\begin{enumerate}[leftmargin=*]
  \item Tum $y$ terimlerini bir tarafa, $x$ terimlerini diger tarafa topla:
    \[
    \frac{1}{g(y)} \, dy = f(x) \, dx.
    \]
  \item Her iki tarafi da integre et:
    \[
    \int \frac{1}{g(y)} \, dy = \int f(x) \, dx + C.
    \]
  \item Mumkunse $y$'yi yalniz birak.
\end{enumerate}

\paragraph{Mini ornek.}
\[
\frac{dy}{dx} = x y
\]
Ayir:
\[
\frac{1}{y} \, dy = x \, dx.
\]
Integralle:
\[
\ln|y| = \frac{x^2}{2} + C \quad \Rightarrow \quad y = C e^{x^2/2}.
\]

Bu tip denklemler, temel seviye fizik ve uygulama sorularinda sik karsina cikar.

\subsection{Birinci Mertebe Lineer Denklemler}

\textbf{Genel form:}
\[
y' + p(x) y = q(x).
\]

\textbf{Recete (entegrasyon faktoru):}
\begin{enumerate}[leftmargin=*]
  \item $p(x)$'i belirle.
  \item Entegrasyon faktorunu hesapla:
    \[
    \mu(x) = e^{\int p(x) \, dx}.
    \]
  \item Denklemin her iki yanini $\mu(x)$ ile carp:
    \[
    \mu(x) y' + \mu(x) p(x) y = \mu(x) q(x).
    \]
  \item Sol taraf artik bir turevdir:
    \[
    (\mu(x) y)' = \mu(x) q(x).
    \]
  \item Her iki tarafi integra et:
    \[
    \mu(x) y = \int \mu(x) q(x) \, dx + C.
    \]
  \item Son olarak $y$'yi yalniz birak:
    \[
    y(x) = \frac{1}{\mu(x)}\left( \int \mu(x) q(x) \, dx + C \right).
    \]
\end{enumerate}

\paragraph{Not.} Ana dosyanin turev-integral bolumunde bu tipe ait detayli bir ornek var.

\subsection{Sabit Katsayili 2. Mertebe Lineer ODE'ler}

\textbf{Genel form:}
\[
ay'' + by' + cy = 0, \quad a \ne 0.
\]

\textbf{Adim 1: Karakteristik denklem.}
\[
a\lambda^2 + b\lambda + c = 0.
\]

Cozumler kok turune gore ayrilir.

\subsubsection{Iki Gercek Ayri k Kok (iki farkli gercek kok)}

Karakteristik denklemden iki farkli gercek kok elde edildigini varsayalim; bunlari
$\lambda_1$ ve $\lambda_2$ ile gosterirsek genel cozum
\[
y(x) = C_1 e^{\lambda_1 x} + C_2 e^{\lambda_2 x}
\]
seklindedir.

\subsubsection{Cakisik Gercek Kok (cakisik gercek kok)}

Eger karakteristik denklemden tek bir gercek kok (cakisik kok) cikarsa, bu koku
$\lambda$ ile gosteririz ve genel cozum
\[
y(x) = (C_1 + C_2 x) e^{\lambda x}
\]
seklindedir.

\subsubsection{Karmasik Esl enik Kokler (alfa artı-eksi i beta turu karmasik kokler)}

Karakteristik denklemden karmasik eslenik kokler elde edersek, bunlari
$\lambda_{1,2} = \alpha \pm i\beta$ seklinde yazabiliriz. Bu durumda genel cozum
\[
y(x) = e^{\alpha x} (C_1 \cos(\beta x) + C_2 \sin(\beta x))
\]
bi\c cimindedir.

\paragraph{Ozet.}
\begin{itemize}[leftmargin=*]
  \item $\Delta = b^2 - 4ac > 0$ ise: iki gercek ayri k kok.
  \item $\Delta = 0$ ise: cakisik kok.
  \item $\Delta < 0$ ise: karmasik kokler.
\end{itemize}

\subsection{Zorlanmis (Sag Tarafli) Sabit Katsayili Lineer ODE}

\textbf{Form:}
\[
ay'' + by' + cy = g(x).
\]

\textbf{Genel strateji:}
\begin{enumerate}[leftmargin=*]
  \item Once homojen kismi coz: $ay'' + by' + cy = 0$ \,$\Rightarrow$\, $y_h$.
  \item Sonra, $g(x)$'in turune gore bir \emph{ozel cozum} $y_p$ tahmin et:
    \begin{itemize}[leftmargin=*]
      \item $g(x)$ polinom ise: polinom formunda dene.
      \item $g(x) = e^{kx}$ ise: $Ae^{kx}$ dene.
      \item $g(x) = \sin bx$ veya $\cos bx$ ise: $A\cos bx + B\sin bx$ dene.
    \end{itemize}
  \item Toplam cozum: $y = y_h + y_p$.
\end{enumerate}

Detayli hesap, ders seviyene gore degisebilir; fakat bu \emph{iskelet} hep aynidir.

\subsection{Fiziksel Yorumlar (Cok Kisa)}

\begin{itemize}[leftmargin=*]
  \item \textbf{Harmonik osilator:} $y'' + \omega^2 y = 0$  $\Rightarrow$ sin\"us-kosin\"us tipinde salinim cozumleri.
  \item \textbf{Sonumlu salinim:} $y'' + 2\gamma y' + \omega^2 y = 0$  
    icin koklerin dogasi (gercek/cakisik/karmasik) sistemin ne kadar hizli sonumlend igini belirler.
\end{itemize}

Bu tur yorumlar, elde ettigin cozumun formunun fiziksel olarak mantikli olup
olmadigini kontrol etmek icin iyi bir sezgi kaynagi saglar.

\section*{Diferansiyel Denklem Ornekleri ve Alistirmalar}

Asagidaki kisa ornekler, farkli ODE turleri icin receteleri uygulama alistirmasi
olarak dusunulebilir. Cogu icin sadece genel cozumu bulman yeterlidir.

\begin{enumerate}[leftmargin=*]
  \item $\displaystyle \frac{dy}{dx} = 3y$ icin ayrilabilirlik recetesini kullanarak genel cozumu bul.
  \item $\displaystyle \frac{dy}{dx} = x y$ denklemi icin ayirma adimlarini yaz ve genel cozumu elde et.
  \item $\displaystyle y' + y = 0$ denklemini birinci mertebe lineer receteyle coz.
  \item $\displaystyle y' - 2y = e^{3x}$ denklemi icin integrasyon faktorunu yaz ve genel cozumu bul.
    \textbf{Cozum:} Entegrasyon faktoru $\mu(x) = e^{\int -2 dx} = e^{-2x}$.
    Denklemi carparsak: $(e^{-2x}y)' = e^{-2x}e^{3x} = e^x$.
    Integral alirsak: $e^{-2x}y = \int e^x dx = e^x + C$.
    Genel cozum: $y(x) = e^{3x} + Ce^{2x}$.
  \item $y'' - 3y' + 2y = 0$ icin karakteristik denklemi yaz, kokleri bul ve genel cozumu belirt.
  \item $y'' + y = 0$ denklemi icin cozumun sin ve cos cinsinden genel formunu yaz.
  \item $y'' + 4y = 0$ denkleminin karakteristik koklerini ve buna karsilik gelen salinim
    frekansini (omega) yorumla.
  \item $y'' - y = e^x$ gibi zorlanmis bir denklem icin (homojen + ozel cozum) genel cozum
    stratejisini kisa maddeler halinde ozetle (detayli hesap zorunlu degil).
  \item Baslangic deger problemi: $y' = 2y$, $y(0) = 3$ icin once genel cozumu bul, sonra baslangic
    kosulunu kullanarak sabiti belirle.
  \item $y'' + 2y' + y = 0$ denklemi icin $\Delta = b^2 - 4ac$ degerini hesapla ve koklerin hangi
    tipte oldugunu (ayrik, cakisik, karmasik) siniflandir; buna gore genel cozumu yaz.
\end{enumerate}
