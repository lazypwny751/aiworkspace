% Temel Fonksiyonlar ve Grafik Sezgisi Modul Dosyasi

\section{Temel Fonksiyonlar ve Grafik Sezgisi}

Bu bolumde, diferansiyel denklemler dersinde surekli karsilasacagin \emph{temel fonksiyon tipleri}
ve bunlarin grafik/sezgi acisindan davranislari biraz daha detayli ozetlenmistir. Polinomlar bir
onceki bolumde ele alindigi icin, burada daha cok diger sik fonksiyon ailelerine odaklaniyoruz.

\subsection{Us, Kok ve Rasyonel Fonksiyonlar}

\paragraph{Us fonksiyonlari.} \(f(x) = x^n\) seklindeki fonksiyonlar icin:
\begin{itemize}
  \item \(n\) tek ise grafik orijinden gecen S-sekilli (\(x^3\) gibi) veya diagonal (\(x\) gibi) bir sekil cizer.
  \item \(n\) cift ise grafik orijinden gecmez, genelde \(y\)-ekseni etrafinda simetriktir (\(x^2, x^4\) gibi).
\end{itemize}

\paragraph{Mini ornek.} \(f(x)=x^2\) icin \(f(-1)=1\), \(f(0)=0\), \(f(1)=1\); grafik yukari acilan bir paraboldu r.
\(f(x)=x^3\) icin \(f(-1)=-1\), \(f(0)=0\), \(f(1)=1\); grafik orijinden gecen S-seklindedir.

\paragraph{Kok fonksiyonlari.} \(f(x) = \sqrt[m]{x}\) gibi fonksiyonlar, genelde sadece belli
bir aralikta tanimlidir (mesela \(\sqrt{x}\) icin \(x \ge 0\)) ve grafik orijinden cikip yavas
sekilde artar.

\paragraph{Rasyonel fonksiyonlar.} \(f(x) = \dfrac{P(x)}{Q(x)}\) bicimindeki fonksiyonlar, payda
sifir oldugunda tanimsizdir ve bu noktalarda \emph{dikey asimptot} ortaya cikabilir.
Pay ve payda derecelerine gore sonsuzdaki davranis (yatay/asimptotik davranis) belirlenir.

\paragraph{Mini ornek.} \(f(x)=\dfrac{1}{x-1}\) icin \(x=1\) noktasinda tanimsizlik ve dikey asimptot vardir;
\(x\to 1^-\) icin \(f(x)\to -\infty\), \(x\to 1^+\) icin \(f(x)\to +\infty\).

\subsection{Ustel ve Logaritmik Fonksiyonlar}

\paragraph{Ustel fonksiyonlar.} \(f(x) = a^x\) (\(a>0, a\neq 1\)) icin:
\begin{itemize}
  \item Tanım araligi genelde \(\mathbb{R}\), deger araligi \((0,\infty)\) dir.
  \item \(a>1\) ise grafik artan; \(0<a<1\) ise azalan seklindedir.
  \item Diferansiyel denklemlerde, \(e^{kx}\) turu cozumler cok sik karsina cikar.
\end{itemize}

\paragraph{Mini ornek.} \(f(x) = 2^x\) icin \(f(0)=1\), \(f(1)=2\), \(f(2)=4\); \(x\to -\infty\) giderken \(f(x)\to 0\).

\paragraph{Logaritmik fonksiyonlar.} \(f(x) = \log_a x\) icin:
\begin{itemize}
  \item Tanım araligi \((0,\infty)\), deger araligi \(\mathbb{R}\) dir.
  \item \(a>1\) ise artan, \(0<a<1\) ise azalan fonksiyondur.
  \item \(y = \log_a x\), \(y = a^x\)'in ters fonksiyonudur.
\end{itemize}

\paragraph{Mini ornek.} \(f(x)=\ln x\) icin \(f(1)=0\), \(f(e)=1\) ve \(x\to 0^+\) giderken \(f(x)\to -\infty\).

\subsection{Trigonometrik Fonksiyonlar}

\paragraph{Temel trig fonksiyonlari.} \(\sin x, \cos x, \tan x\) ve bunlarin turevleri,
cozduklerin diferansiyel denklemlerde sik cikar:
\begin{itemize}
  \item \(\sin x\) ve \(\cos x\): periyodik, deger araligi \([-1,1]\).
  \item \(\tan x\): periyodik, dikey asimptotlari olan bir fonksiyondur.
  \item \(\sin x\)'in turevi \(\cos x\), \(\cos x\)'in turevi \(-\sin x\)'tir.
\end{itemize}

\paragraph{Mini ornek.} \(f(x)=\sin x\) icin \(f(0)=0\), \(f(\tfrac{\pi}{2})=1\), \(f(\pi)=0\).
Bir tam periyot uzunlugu \(2\pi\)'dir.

Bu fonksiyonlar, salınım (osilasyon) iceren cozumlerde (mesela ikinci mertebe sabit katsayili
lineer ODE'lerde) dogrudan karsina cikar.

\subsection{Grafik ve Sezgi Odakli Notlar}

Farkli fonksiyon tiplerinin grafiklerini kafanda netlestirmek, hem limit/turev yorumunu
hem de diferansiyel denklem cozumlerinin davranisini anlamani cok kolaylastirir.

\begin{itemize}
  \item \textbf{Artan/azalan}: Grafik sag tarafa giderken yukari mi cikiyor, asagi mi iniyor?
  \item \textbf{Konveks/konkav}: Grafikteki "cukur" ve "tepe" hissi, ikinci turev isaretine baglidir.
  \item \textbf{Asimptotlar}: Rasyonel veya logaritmik fonksiyonlarda, grafigin
        yaklastigi ama hic dokunmadigi dogrular var mi?
\end{itemize}

\paragraph{Hizli tekrar icin mini tablo.}
\begin{center}
\begin{tabular}{|c|c|c|c|}
  \hline
  Fonksiyon & Turev & Temel ozellik & Not \\
  \hline
  $x^n$ & $nx^{n-1}$ & polinom, her yerde duzgun & $n$ tamsayi \\
  $e^x$ & $e^x$ & hep pozitif, hizla artan & cozumlerde cok sik \\
  $\ln x$ & $1/x$ & $x>0$, yavas artan & $x\to 0^+$ icin $-\infty$ \\
  $\sin x$ & $\cos x$ & periyodik, sinirli & osilasyon \\
  $\cos x$ & $-\sin x$ & periyodik, sinirli & faz kaymali sin \\
  \hline
\end{tabular}
\end{center}

Bu bolumu, polinomlar ve limit-sureklilik bolumleri ile birlikte dusunebilirsin:
\begin{itemize}
  \item Once \emph{hangi fonksiyonla ugrastigini} tanimla (polinom, us, rasyonel, trig vs.).
  \item Sonra limit/turev/integral sorusunda buna gore davranisi yorumla.
  \item Son olarak, diferansiyel denklem cozumlerinin de bu temel fonksiyonlardan kombinasyonlar
        oldugunu hatirla.
\end{itemize}

\section*{Temel Fonksiyon Ornekleri ve Alistirmalar}

Aşağıdaki kisa sorular, farkli fonksiyon tiplerinin tanim kumesi, degerleri ve temel
grafik davranisini pekistirmek icin hazirlanmistir.

\begin{enumerate}[leftmargin=*]
  \item $f(x) = 2x^2 - 3x + 1$ icin $f(-1)$, $f(0)$ ve $f(2)$ degerlerini hesapla.\\
		extbf{Sonuc:} $f(-1)=2\cdot 1+3+1=6$, $f(0)=1$, $f(2)=2\cdot 4-6+1=3$.

  \item $f(x) = \dfrac{1}{x-2}$ fonksiyonunun tanim kumesini ve $x=2$ yakinindaki
    davranisini (limitleri) yorumla.\\
		extbf{Sonuc:} Tanim kumesi $x\neq 2$; $x\to 2^-$ icin $f(x)\to -\infty$, $x\to 2^+$ icin $f(x)\to +\infty$.

  \item $f(x) = 2^x$ icin $f(0)$, $f(1)$, $f(2)$ ve $f(-1)$ degerlerini bul; $x\to -\infty$
    giderken fonksiyonun neye yaklastigini soyle.\\
		extbf{Sonuc:} $f(0)=1$, $f(1)=2$, $f(2)=4$, $f(-1)=1/2$; $x\to -\infty$ icin $f(x)\to 0$.

  \item $f(x) = 3^x$ fonksiyonunun $x$ arttikca nasil davrandigini kisaca acikla (artan/azalan,
    alt siniri var mi?).\\
		extbf{Sonuc:} $3^x$ artan bir fonksiyondur, alt siniri 0'dir (asla 0 olmaz, 0'a yaklasir).

  \item $f(x) = \ln x$ fonksiyonunun tanim kumesini, $x\to 0^+$ ve $x\to \infty$ icin
    limit davranisini belirt.\\
		extbf{Sonuc:} Tanim kumesi $(0,\infty)$; $x\to 0^+$ icin $\ln x\to -\infty$, $x\to \infty$ icin $\ln x\to \infty$.

  \item $f(x) = \sqrt{x-1}$ icin tanim kumesini bul ve $x$ karsilikli birkac deger icin
    fonksiyon degerlerini hesapla (mesela $x=1,2,5$).\\
		extbf{Sonuc:} Tanim kumesi $x\ge 1$; $f(1)=0$, $f(2)=1$, $f(5)=2$.

  \item $f(x) = \sin x$ icin $x=0, \tfrac{\pi}{2}, \pi, \tfrac{3\pi}{2}$ noktalarinda
    degerleri bul ve bir periyot boyunca grafigin nasil gectigini tarif et.\\
		extbf{Sonuc:} $f(0)=0$, $f(\tfrac{\pi}{2})=1$, $f(\pi)=0$, $f(\tfrac{3\pi}{2})=-1$;
        grafik bir periyotta 0'dan 1'e cikip tekrar 0'a iner ve -1'den yine 0'a doner.

  \item $f(x) = \cos x$ icin bir periyot icerisinde maksimum, minimum ve sifir oldugu
    noktalarin bir listesini yap.\\
		extbf{Sonuc:} Bir periyotta $[0,2\pi]$ icin: maksimum $1$ degeri $x=0,2\pi$'de;
        minimum $-1$ degeri $x=\pi$'de; sifir oldugu noktalar $x=\tfrac{\pi}{2},\tfrac{3\pi}{2}$.

  \item $f(x) = \tan x$ fonksiyonunun tanimsiz oldugu noktalar icin genel bir formul yaz
    (kacinci katlarin \(\tfrac{\pi}{2}\) oldugunu belirt) ve bu noktalarda grafigin
    nasil davrandigini acikla.\\
		extbf{Sonuc:} Tanim siz oldugu noktalar $x = \tfrac{\pi}{2} + k\pi$ (her tamsayi $k$ icin);
        bu noktalarda dikey asimptot vardir, grafik bir tarafta $+\infty$, diger tarafta $-\infty$'ye gider.

  \item Ayni koordinat duzlemine kabaca $y = e^x$ ve $y = \ln x$ grafigini cizdigini hayal et:
    hangi fonksiyon hangi bolgede daha buyuk, nerede kesisiyorlar (yaklasik)?\\
		extbf{Sonuc:} $y=e^x$ her zaman $y=\ln x$'ten daha buyuktur; bunlar aslinda ters fonksiyon
        ciftidir ve $y=x$ dogrusu uzerinde ayna simetrisine sahiptir.
\end{enumerate}
