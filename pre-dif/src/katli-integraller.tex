% Katli Integraller Modul Dosyasi

\section{Katli Integraller}

Bu bolum, iki ve uc katli integralleri \emph{recete gibi} hatirlamak icin kisa bir ozet sunar.

\subsection{Iki Katli Integral: Temel Fikir}

Suresiz ve uygun sekilde duzenli bir \(f(x,y)\) fonksiyonu icin, dikdortgensel bir bolge uzerinde
\(\iint\) su sekilde tanimlanir. \(R = [a,b] \times [c,d]\) olsun. O zaman
\begin{align*}
  \iint_R f(x,y)\,\mathrm{d}A
  &= \int_a^b \int_c^d f(x,y)\,\mathrm{d}y\,\mathrm{d}x \\
  &= \int_c^d \int_a^b f(x,y)\,\mathrm{d}x\,\mathrm{d}y.
\end{align*}

Genelde, hangi siralama kolay geliyorsa onu secersin; ama \emph{sinirlarin dogru yazilmasi} kritik nokta.

\paragraph{Hesaplanmis ornek (dikdortgen bolge).}
\[
  \int_0^1 \int_0^2 (x + y)\,\mathrm{d}y\,\mathrm{d}x.
\]
Once ic integrali (\(y\) ye gore) al:
\begin{align*}
  \int_0^2 (x+y)\,\mathrm{d}y
  &= \left[ xy + \tfrac{1}{2}y^2 \right]_{y=0}^{y=2}
   = 2x + 2.
\end{align*}
Sonra dis integrali (\(x\) e gore) al:
\begin{align*}
  \int_0^1 (2x+2)\,\mathrm{d}x
  &= \left[ x^2 + 2x \right]_0^1 = 1 + 2 = 3.
\end{align*}
Yani iki katli integralin sonucu \(3\)'tur.

\subsection{Dikdortgen Olmayan Bolgeler}

Pratikte siklikla asagidaki tip bolgelerle karsilasirsin:
\begin{itemize}
  \item \(D = \{(x,y) : a \le x \le b,\ g_1(x) \le y \le g_2(x)\}\)
  \item \(D = \{(x,y) : c \le y \le d,\ h_1(y) \le x \le h_2(y)\}\)
\end{itemize}

Ilk tip icin
\[
  \iint_D f(x,y)\,\mathrm{d}A = \int_a^b \left( \int_{g_1(x)}^{g_2(x)} f(x,y)\,\mathrm{d}y \right)\mathrm{d}x,
\]
ikinci tip icin benzer sekilde once \(x\), sonra \(y\) uzerinden integral alirsin.

\paragraph{Pratik recete.}
\begin{enumerate}[label=\arabic*)]
  \item Bolgeyi mutlaka kabaca da olsa ciz.
  \item \(x\) veya \(y\)'ye gore bir \emph{kesit} al: alt ve ust sinir fonksiyonlarini netlestir.
  \item Integral siralamasini bu kesite gore yaz (\(y\) icte, \(x\) diste ya da tam tersi).
\end{enumerate}

\subsection{Koordinat Donusumleri (Ozet)}

Bazi katli integraller, uygun bir koordinat donusumu ile cok daha kolay hale gelir.
En temel ornek \textbf{polar (kutupsal) koordinatlar}dir.

\subsubsection*{Polar koordinatlar}

\[
  x = r \cos\theta, \qquad y = r \sin\theta.
\]

Bu donusumde alan elemani
\[
  \mathrm{d}A = r\,\mathrm{d}r\,\mathrm{d}\theta
\]
seklinde gelir; yani integrale ekstra bir \(r\) carpani eklenir.
Genel form:
\[
  \iint_D f(x,y)\,\mathrm{d}A
  = \int_{\theta_1}^{\theta_2} \int_{r_1(\theta)}^{r_2(\theta)}
      f(r\cos\theta, r\sin\theta)\, r\,\mathrm{d}r\,\mathrm{d}\theta.
\]

Daire, disk, halka gibi dairesel simetrili bolgelerde polar koordinat secimi neredeyse her zaman isin kolaylastirir.

\paragraph{Hesaplanmis ornek (disk uzerinde integral).}

\(D\), merkezde yaricapi 1 olan birim disk olsun: \(x^2 + y^2 \le 1\). \(f(x,y)=x^2 + y^2\)
icin
\[
  \iint_D (x^2 + y^2)\,\mathrm{d}A
\]
degerini hesaplayalim.

Polar koordinatlara gecersek \(x^2 + y^2 = r^2\) ve \(\mathrm{d}A = r\,\mathrm{d}r\,\mathrm{d}\theta\) oldugundan,
bolge \(0 \le r \le 1\), \(0 \le \theta \le 2\pi\) ile tarif edilir:
\begin{align*}
  \iint_D (x^2 + y^2)\,\mathrm{d}A
  &= \int_0^{2\pi} \int_0^1 r^2 \cdot r\,\mathrm{d}r\,\mathrm{d}\theta \\
  &= \int_0^{2\pi} \int_0^1 r^3\,\mathrm{d}r\,\mathrm{d}\theta \\
  &= \int_0^{2\pi} \left[ \tfrac{1}{4} r^4 \right]_0^1 \mathrm{d}\theta \\
  &= \int_0^{2\pi} \tfrac{1}{4}\,\mathrm{d}\theta \\
  &= \tfrac{1}{4} \cdot 2\pi = \frac{\pi}{2}.
\end{align*}

\subsection{Uc Katli Integraller}

Uc degiskenli bir \(f(x,y,z)\) fonksiyonu icin
\[
  \iiint_E f(x,y,z)\,\mathrm{d}V
\]
ifadesi, hacim uzerinde integral anlamina gelir. Dikdortgensel bir bolge icin
\[
  E = [a,b] \times [c,d] \times [e,f]
\]
oldugunda, integral ic ice uc tekli integral seklinde yazilir:
\[
  \iiint_E f(x,y,z)\,\mathrm{d}V
  = \int_a^b \int_c^d \int_e^f f(x,y,z)\,\mathrm{d}z\,\mathrm{d}y\,\mathrm{d}x,
\]
veya degisken siralamasi ihtiyaca gore degistirilebilir.

\subsection{Diferansiyel Denklemlerle Baglanti}

Baslangic diferansiyel denklemleri dersinde katli integraller cok sik sorulmayabilir;
ancak daha ileri konularda ve fiziksel modellere gecince:
\begin{itemize}
  \item Hacim/alan hesaplari,
  \item Yogunluk fonksiyonu ile \emph{toplam kutle},
  \item Ortalama deger hesaplari
\end{itemize}
hep katli integral fikrine dayanir.

Ote yandan, PDE (kismen diferansiyel denklemler) tarafinda sinir deger problemleri,
enerji/isi korunumu gibi konulara girdiginde, bu sayfaya geri donup
\emph{"bolge uzerinde integral nasil yaziliyordu"} diye bakmak isini kolaylastirir.

\section*{Katli Integral Ornekleri ve Alistirmalar}

Asagidaki ornekler, iki katli ve uc katli integrallerin kurulum ve hesaplamasini
pratik etmen icin hazirlanmistir.

\begin{enumerate}[leftmargin=*]
  \item $\displaystyle \int_0^1 \int_0^1 (x + 2y)\,\mathrm{d}y\,\mathrm{d}x$ integralini hesapla.\\
        	extbf{Sonuc:} Ic integral $\left[xy + y^2\right]_0^1 = x+1$;
        dis integral $\int_0^1 (x+1)\,dx = \left[\tfrac{x^2}{2} + x\right]_0^1 = \tfrac{3}{2}$.

  \item $\displaystyle \int_0^2 \int_0^1 (3x - y)\,\mathrm{d}x\,\mathrm{d}y$ icin once ic, sonra dis
    integrali alarak sonucu bul.\\
        	extbf{Sonuc:} Ic integral $\left[\tfrac{3}{2}x^2 - yx\right]_0^2 = 6-2y$;
        dis integral $\int_0^1 (6-2y)\,dy = [6y - y^2]_0^1 = 5$.

  \item $\displaystyle \int_0^1 \int_0^2 (x^2 + y)\,\mathrm{d}y\,\mathrm{d}x$ integralini hesaplarken,
    ister $y$'ye gore ister $x$'e gore once integre etmeyi dene (sonuc ayni olacak).\\
        	extbf{Sonuc:} $\displaystyle \int_0^1 \left[x^2 y + \tfrac{y^2}{2}\right]_0^2 dx = \int_0^1 (2x^2 + 2) dx = \left[\tfrac{2}{3}x^3 + 2x\right]_0^1 = \tfrac{8}{3}$.

  \item $D = [0,1] \times [0,3]$ icin $\displaystyle \iint_D (2x + y)\,\mathrm{d}A$ integralini
    acik integral seklinde yaz ve hesapla.\\
        	extbf{Sonuc:} $\displaystyle \int_0^1 \int_0^3 (2x+y)\,dy\,dx = \int_0^1 (6x+\tfrac{9}{2}) dx
        = \left[3x^2 + \tfrac{9}{2}x\right]_0^1 = \tfrac{15}{2}$.

  \item $D = \{(x,y) : 0 \le x \le 1,\ x \le y \le 1\}$ bolgesi icin
    $\displaystyle \iint_D y\,\mathrm{d}A$ integralini once $y$ sonra $x$ uzerinden yaz.\\
            extbf{Sonuc:} $\displaystyle \int_0^1 \int_x^1 y\,dy\,dx = \int_0^1 \left[\tfrac{y^2}{2}\right]_x^1 dx
        = \int_0^1 \left(\tfrac{1}{2} - \tfrac{x^2}{2}\right) dx = \left[\tfrac{x}{2} - \tfrac{x^3}{6}\right]_0^1 = \tfrac{1}{3}$.

  \item Yukaridaki $D$ bolgesi icin ayni integrali bu kez once $x$, sonra $y$ uzerinden
    yazmayi dene (limitleri degisken fonksiyonlar seklinde ifade et).\\
          extbf{Sonuc:} $D$ icin $0 \le y \le 1$, $0 \le x \le y$; integral $\displaystyle \int_0^1 \int_0^y y\,dx\,dy
  = \int_0^1 y^2 \,dy = \left[\tfrac{y^3}{3}\right]_0^1 = \tfrac{1}{3}$ (ayni sonuc).

  \item Birim kare $[0,1]\times[0,1]$ uzerinde $f(x,y)=x^2y$ icin ortalama deger
    $\displaystyle f_{ort} = \frac{1}{\text{alan}} \iint f\,\mathrm{d}A$ formuluyle hesapla.\\
          extbf{Sonuc:} Alan 1 oldugu icin $f_{ort} = \displaystyle \int_0^1\int_0^1 x^2 y\,dy\,dx
  = \int_0^1 x^2 \left[\tfrac{y^2}{2}\right]_0^1 dx = \int_0^1 \tfrac{x^2}{2} \,dx = \tfrac{1}{6}$.

  \item $E = [0,1]\times[0,1]\times[0,1]$ icin
    $\displaystyle \iiint_E (x + y + z)\,\mathrm{d}V$ hacim integralini hesapla.\\
          extbf{Sonuc:} Ayrilabilir: $\displaystyle \iiint_E (x+y+z)\,dV = \int_0^1 x\,dx + \int_0^1 y\,dy + \int_0^1 z\,dz
  = 3 \cdot \tfrac{1}{2} = \tfrac{3}{2}$.

  \item Polar koordinatlara gecerek, yaricapi 2 olan diskte
    $\displaystyle \iint_D x^2 \,\mathrm{d}A$ integralini kur (hesaplamayi istersen tam
    yap, istersen kurulmus sekilde birak).\\
          extbf{Sonuc:} $x = r\cos\theta$, $\mathrm{d}A = r\,dr\,d\theta$; integrand $x^2 = r^2 \cos^2\theta$;
  integral $\displaystyle \int_0^{2\pi} \int_0^2 r^3 \cos^2\theta\,dr\,d\theta$ seklinde yazilir.

  \item Yine polar koordinatlarla, $\displaystyle \iint_D 1\,\mathrm{d}A$ integralinin aslinda
    bolgenin alanini verdigini hatirla ve yaricapi $R$ olan bir diskin alanini bu sekilde bul.\\
          extbf{Sonuc:} $\displaystyle \int_0^{2\pi} \int_0^R 1\cdot r\,dr\,d\theta = \int_0^{2\pi} \left[\tfrac{r^2}{2}\right]_0^R d\theta
  = \int_0^{2\pi} \tfrac{R^2}{2} \,d\theta = \pi R^2$.
\end{enumerate}
